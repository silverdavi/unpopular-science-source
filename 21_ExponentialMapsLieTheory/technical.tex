% Custom semantic coloring
\newcommand{\A}[1]{\textcolor{green!80!black}{#1}}          % Elements from domain A
\newcommand{\B}[1]{\textcolor{orange!80!black}{#1}} % Elements from codomain B
\newcommand{\F}[1]{\textcolor{blue!60!black}{#1}}   % Scalars from field F
\newcommand{\OpA}[1]{\textcolor{green!70!black}{#1}}        % Operations/Identities in A
\newcommand{\OpB}[1]{\textcolor{orange!70!black}{#1}} % Operations/Identities in B

\begin{technical}
{\Large\textbf{Exponentials from Derivations}}\\[0.5em]
\textit{Why \( f' = \lambda f \) implies \( f(x+y) = f(x)f(y) \)}

\vspace{0.5em}
\noindent\textbf{1. Functional Characterization via Derivation}\\[-0.5em]

Let \( (\A{A}, \OpA{+}, \OpA{0}) \) be an additive group and \( (\B{B}, \OpB{+}, \OpB{\cdot}, \OpB{0}, \OpB{1}) \) a unital commutative ring. Let \( \mathrm{Func}(\A{A}, \B{B}) \) be the ring of functions with pointwise operations.

Define shift operator \( T_{\A{y}}: \mathrm{Func}(\A{A}, \B{B}) \to \mathrm{Func}(\A{A}, \B{B}) \) by \( (T_{\A{y}}g)(\A{x}) := g(\A{x} \OpA{+} \A{y}) \).

Suppose \( K: \mathrm{Func}(\A{A}, \B{B}) \to \mathrm{Func}(\A{A}, \B{B}) \) satisfies:

\vspace{-0.5em}
\begin{itemize}[leftmargin=3em,topsep=0pt,itemsep=0pt]
  \item[(A)] Additivity: \( K(g \OpB{+} h) = K(g) \OpB{+} K(h) \)
  \item[(L)] Leibniz Rule: \( K(g \OpB{\cdot} h) = K(g) \OpB{\cdot} h \OpB{+} g \OpB{\cdot} K(h) \)
  \item[(C)] Kills Constants: \( K(c_{\B{b}}) = c_{\OpB{0}} \) where \( c_{\B{b}}(\A{x}) \equiv \B{b} \)
  \item[(T)] Translation Invariance: \( K \circ T_{\A{y}} = T_{\A{y}} \circ K \)
\end{itemize}

Let \( f \in \mathrm{Func}(\A{A}, \B{B}) \) and \( \B{\lambda} \in \B{B} \) satisfy:

\vspace{-0.5em}
\begin{itemize}[leftmargin=3em,topsep=0pt,itemsep=0pt]
  \item[(E)] Eigenfunction: \( K(f) = \B{\lambda} \OpB{\cdot} f \)
  \item[(N)] Normalization: \( f(\OpA{0}) = \OpB{1} \)
  \item[(U)] Uniqueness: \( K(g) = \B{\lambda} \OpB{\cdot} g \), \( g(\OpA{0}) = \OpB{0} \)\\
    \( \Rightarrow g \equiv \OpB{0} \)
\end{itemize}

Note that (T) implies shifted eigenfunctions remain eigenfunctions. For \( f \) satisfying (E), let \( g_{\A{y}} := T_{\A{y}}f \). Then:
\begin{align*}
K(g_{\A{y}}) &= K(T_{\A{y}}f) \notag\\
&= T_{\A{y}}(K(f)) \notag\\
&= T_{\A{y}}(\B{\lambda} \OpB{\cdot} f) \notag\\
&= \B{\lambda} \OpB{\cdot} T_{\A{y}}(f) \notag\\
&= \B{\lambda} \OpB{\cdot} g_{\A{y}} \notag
\end{align*}

Now define \( g(\A{x}) := f(\A{x} \OpA{+} \A{y}) \OpB{-} f(\A{x}) \OpB{\cdot} f(\A{y}) = g_{\A{y}}(\A{x}) \OpB{-} f(\A{x}) \OpB{\cdot} f(\A{y}) \).\\
Then \( g(\OpA{0}) = f(\A{y}) \OpB{-} \OpB{1} \OpB{\cdot} f(\A{y}) = \OpB{0} \).

Computing \( K(g) \) using (A), (L), and (C) (with \( c_{f(\A{y})} \) the constant function):
\begin{align*}
K(g) &= K(g_{\A{y}}) \OpB{-} K(f \OpB{\cdot} c_{f(\A{y})}) \notag\\
&= K(g_{\A{y}}) \OpB{-} (K(f) \OpB{\cdot} c_{f(\A{y})} \OpB{+} f \OpB{\cdot} K(c_{f(\A{y})})) \notag\\
&= \B{\lambda} \OpB{\cdot} g_{\A{y}} \OpB{-} (\B{\lambda} \OpB{\cdot} f \OpB{\cdot} c_{f(\A{y})} \OpB{+} f \OpB{\cdot} c_{\OpB{0}}) \notag\\
&= \B{\lambda} \OpB{\cdot} g_{\A{y}} \OpB{-} \B{\lambda} \OpB{\cdot} f \OpB{\cdot} c_{f(\A{y})} \notag\\
&= \B{\lambda} \OpB{\cdot} (g_{\A{y}}(\A{x}) \OpB{-} f(\A{x}) \OpB{\cdot} f(\A{y})) \notag\\
&= \B{\lambda} \OpB{\cdot} g(\A{x}) \notag
\end{align*}

Since \( K(g) = \B{\lambda} \OpB{\cdot} g \) and \( g(\OpA{0}) = \OpB{0} \), uniqueness (U) gives \( g \equiv \OpB{0} \). Thus:
\[
f(\A{x} \OpA{+} \A{y}) = f(\A{x}) \OpB{\cdot} f(\A{y})
\]

Hence the exponential property emerges from pure structure: \textbf{(A)} additivity, \textbf{(L)} derivation, \textbf{(T)} translational symmetry, and \textbf{(U)} irreducibility. The functional equation \( f(x+y) = f(x)f(y) \) is a geometric necessity, of which the real exponential \( e^{x+y} = e^x e^y \) is a special case.

\textit{Remark}: If (U) fails, we obtain \( f(\A{x} \OpA{+} \A{y}) = f(\A{x}) \OpB{\cdot} f(\A{y}) \OpB{+} h_{\A{y}}(\A{x}) \) where \( h_{\A{y}} \) lies in \( V_{\B{\lambda}}^{\OpA{0}} := \{g : K(g) = \B{\lambda} \OpB{\cdot} g, g(\OpA{0}) = \OpB{0}\} \). This error term satisfies a cocycle condition central to group cohomology, measuring "how far" \( f \) is from being a homomorphism.

\vspace{0.5em}
\noindent\textbf{2. Geometric Interpretation in Manifolds}\\[-0.5em]

For Riemannian manifold \( M \), \( p \in M \), the exponential map \( \exp_p(v) := \gamma_v(1) \) where \( \gamma_v \) solves:
\[ \frac{D}{dt} \dot{\gamma}_v(t) = 0, \quad \gamma_v(0) = p, \quad \dot{\gamma}_v(0) = v \]

In \( \mathbb{R}^n \): \( \gamma_v(t) = p + tv \), so \( \exp_p(v) = p + v \).\\
In Lie group \( G \subset \mathrm{GL}_n(\mathbb{R}) \):\\
\( \gamma_X(t) = \exp(tX) \) satisfies \( \frac{d}{dt}\gamma(t) = X\gamma(t) \).

The exponential map lifts linear generators to\\
integrated flows across groups/manifolds.

\vspace{0.3em}
\noindent\textbf{References:}\\
{\footnotesize
Lang (2001). \textit{Fund.\ Diff.\ Geom.} Springer.\\
Hall (2015). \textit{Lie Groups, Lie Algebras, and Representations}. Springer.
}
\end{technical}