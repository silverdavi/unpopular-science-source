% Custom semantic coloring
\newcommand{\A}[1]{\textcolor{green!35!black}{#1}}          % Elements from domain A (dark green)
\newcommand{\B}[1]{\textcolor{red!55!black}{#1}} % Elements from codomain B (dark red)
\newcommand{\OpA}[1]{\textcolor{green!45!black}{#1}}        % Operations/Identities in A (dark green)
\newcommand{\OpB}[1]{\textcolor{red!60!black}{#1}} % Operations/Identities in B (dark red)

\begin{technical}
{\Large\textbf{Exponentials}}\\[0.25em]

\noindent\textbf{1. How \( f' = \lambda f \underset{\sim}{\Rightarrow} f(x+y) = f(x)f(y) \)?}\\
Let \( (\A{A}, \OpA{+}, \OpA{0}) \) be an additive group and \( (\B{B}, \OpB{+}, \OpB{\cdot}, \OpB{0}, \OpB{1}) \) a unital commutative ring. Let \( \mathrm{Func}(\A{A}, \B{B}) \) be the ring of functions with pointwise operations.

Define shift operator \( T_{\A{y}}: \mathrm{Func}(\A{A}, \B{B}) \to \mathrm{Func}(\A{A}, \B{B}) \) by \( (T_{\A{y}}g)(\A{x}) := g(\A{x} \OpA{+} \A{y}) \).

Suppose \( K: \mathrm{Func}(\A{A}, \B{B}) \to \mathrm{Func}(\A{A}, \B{B}) \) satisfies:

\vspace{-0.25em}
\begin{itemize}[leftmargin=2em,topsep=0pt,itemsep=0pt]
  \item[(A)] Additivity: \( K(g \OpB{+} h) = K(g) \OpB{+} K(h) \)
  \item[(L)] Leibniz: \( K(g \OpB{\cdot} h) = K(g) \OpB{\cdot} h \OpB{+} g \OpB{\cdot} K(h) \)
  \item[(C)] Kills Constants: \( K(c_{\B{b}}) = c_{\OpB{0}} \) where \( c_{\B{b}}(\A{x}) \equiv \B{b} \)
  \item[(T)] Translation Invariance: \( K \circ T_{\A{y}} = T_{\A{y}} \circ K \)
\end{itemize}

Let \( f \in \mathrm{Func}(\A{A}, \B{B}) \) and \( \B{\lambda} \in \B{B} \) satisfy:

\vspace{-0.25em}
\begin{itemize}[leftmargin=2.5em,topsep=0pt,itemsep=0pt]
  \item[(E)] Eigenfunction: \( K(f) = \B{\lambda} \OpB{\cdot} f \)
  \item[(N)] Normalization: \( f(\OpA{0}) = \OpB{1} \)
  \item[(U)] Uniqueness: Evaluation at \( \OpA{0} \) is injective on \( \ker(K - \B{\lambda} I) \).
\end{itemize}
For \( f \) satisfying (E), let \( g_{\A{y}} := T_{\A{y}}f \). Then:
\begin{align*}
K(g_{\A{y}}) &= K(T_{\A{y}}f) \\
&= T_{\A{y}}(K(f)) = T_{\A{y}}(\B{\lambda} \OpB{\cdot} f) \\
&= \B{\lambda} \OpB{\cdot} T_{\A{y}}(f) = \B{\lambda} \OpB{\cdot} g_{\A{y}}\\
\text{Define } g(\A{x}) &= f(\A{x} \OpA{+} \A{y}) \OpB{-} f(\A{x}) \OpB{\cdot} f(\A{y}) \\
&= g_{\A{y}}(\A{x}) \OpB{-} f(\A{x}) \OpB{\cdot} f(\A{y}),\\
g(\OpA{0}) &= f(\A{y}) \OpB{-} f(\OpA{0}) \OpB{\cdot} f(\A{y}) = \OpB{0}.
\end{align*}
Computing \( K(g) \) using (A), (L), and (C) (with \( c_{f(\A{y})} \) the constant function):
\begin{align*}
K(g) &= K(g_{\A{y}}) \OpB{-} K(f \OpB{\cdot} c_{f(\A{y})}) \\
&= K(g_{\A{y}}) \OpB{-} (K(f) \OpB{\cdot} c_{f(\A{y})} \OpB{+} f \OpB{\cdot} K(c_{f(\A{y})})) \\
&= \B{\lambda} \OpB{\cdot} g_{\A{y}} \OpB{-} (\B{\lambda} \OpB{\cdot} f \OpB{\cdot} c_{f(\A{y})} \OpB{+} f \OpB{\cdot} c_{\OpB{0}}) \\
&= \B{\lambda} \OpB{\cdot} g_{\A{y}} \OpB{-} \B{\lambda} \OpB{\cdot} f \OpB{\cdot} c_{f(\A{y})} \\
&= \B{\lambda} \OpB{\cdot} (g_{\A{y}}(\A{x}) \OpB{-} f(\A{x}) \OpB{\cdot} f(\A{y})) = \B{\lambda} \OpB{\cdot} g(\A{x})
\end{align*}
Since \( K(g) = \B{\lambda} \OpB{\cdot} g \) and \( g(\OpA{0}) = \OpB{0} \), uniqueness (U) gives \( g \equiv \OpB{0} \). Thus:
\vspace{-0.25em}
\begin{align*}
f(\A{x} \OpA{+} \A{y}) = f(\A{x}) \OpB{\cdot} f(\A{y})
\end{align*} 
Hence the additive exponential property emerges from the derivation properties:
\textbf{(A)} additivity, \textbf{(L)} Leibniz, \textbf{(T)} translational symmetry,
and \textbf{(U)} uniqueness at $\OpA{0}$.
The functional equation \( f(x+y) = f(x)f(y) \) is a geometric necessity,
of which the real exponential \( e^{x+y} = e^x e^y \) is a special case.

\textit{Remark}:
If (U) fails, we obtain
{\setlength{\abovedisplayskip}{0.25em}%
 \setlength{\belowdisplayskip}{0.25em}%
 \setlength{\abovedisplayshortskip}{0.2em}%
 \setlength{\belowdisplayshortskip}{0.2em}%
 \begin{align*}
f(\A{x} \OpA{+} \A{y}) &= f(\A{x}) \OpB{\cdot} f(\A{y}) \OpB{+} h_{\A{y}}(\A{x}), \\
h_{\A{y}} &\in V_{\B{\lambda}}^{\OpA{0}} := \{ g : K(g) = \B{\lambda} \OpB{\cdot} g,\ g(\OpA{0}) = \OpB{0} \}.
 \end{align*}}
This error term satisfies a cocycle condition central to group cohomology,
measuring how far \( f \) is from being a homomorphism.

\textit{Intuition}:
Translation invariance makes $K$ commute with shifts: $K(T_{\A{y}} f) = T_{\A{y}} K(f)$. Thus every translate $T_{\A{y}} f$ of an eigenfunction remains in the $\lambda$-eigenspace of $K$. Translation acts on this eigenspace by scalar multiplication: $T_{\A{y}}f = f(\A{y})\,f$. Commuting with the derivation therefore forces addition in the domain (the group operation generating translations) to correspond to multiplication in the codomain (the scalar action on the eigenspace). Eigenfunctions of a translation-invariant derivation therefore turn addition into multiplication.

\vspace{0.2em}
\noindent\textbf{2. Geometric Interpretation in Manifolds}\\
Riemannian exponential: for a Riemannian manifold \( M \) and \( p \in M \),
\( \exp_p(v) := \gamma_v(1) \), where \( \gamma_v \) solves:
\vspace{-0.25em}
\begin{align*}
\frac{D}{dt} \dot{\gamma}_v(t) &= 0,\\
\gamma_v(0) = p&,\ \dot{\gamma}_v(0) = v
\end{align*}

In \( \mathbb{R}^n \): \( \gamma_v(t) = p + tv \), so \( \exp_p(v) = p + v \).\\
Lie group exponential (matrix case): if \( G \subset \mathrm{GL}_n(\mathbb{R}) \) with Lie algebra \( \mathfrak{g} \), the map \( \exp: \mathfrak{g} \to G \) yields one-parameter subgroups
\( \gamma_X(t) = \exp(tX) \) with \( \gamma_X(0) = I \) and \( \gamma_X'(t) = X\,\gamma_X(t) \) (left-invariant convention).
In general, this is not the Riemannian exponential unless the metric is chosen suitably (e.g., bi-invariant metrics).

\vspace{0.25em}
\noindent\textbf{References:}\\
{\footnotesize
Lang (2001). \textit{Fund.\ Diff.\ Geom.} Springer.\\
Hall (2015). \textit{Lie Groups, Lie Algebras, and Representations}. Springer.
}
\end{technical}