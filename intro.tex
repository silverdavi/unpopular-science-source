

This book is not a popular science book. It is not a textbook. It is not an academic book. It is not even a chimera of the above.  

It does share some goals with the three: to inspire wonder (like a popular science book), to include some rigour (like textbooks), and to introduce readers to phenomena that might challenge their understanding (as academic works often achieve).

As with some combinations, like a sushi-pizza restaurant, it excels at none.  
The main exposition isn't long enough for full understanding, the technical part is often too abstract or detailed to follow, and despite claims of rigour, von Neumann's line that \textit{there's no sense in being precise when you don't know what you're talking about} fits too well.

But if my plan works, you'll get the appetite to leave this sushi-pizza diner. Maybe to a textbook because you're intrigued. Maybe to Wikipedia or blogs for more context about this fantastical world.

This book contains 50 stories, each structured to guide readers from the intuitive to the profound:

\textbf{Backdrop} \ Each chapter begins with concise background — the people, circumstances, and discoveries behind the phenomenon.

\textbf{Phenomenon Description} \ The phenomenon is described in straightforward terms, avoiding sensational language for clear, accurate explanations.

\textbf{Hardcore Analysis} \ For readers ready to dive deeper, the third section provides rigorous academic analysis. Here, the mathematical and technical underpinnings of the phenomenon are laid bare, complete with equations, references, and detailed derivations. This section is unapologetically tough. Like references in a scientific article, this section is not required for the reader to grasp the main ideas, but it does provide scaffolding, justifies the clarity above it, and offers readers the tools to validate the claims, explore further, or simply appreciate that simplified versions are built on layers of rigor.

Few disclaimers: The book contains errors ranging from typos to wrong equations. Please report them, and be forgiving of mistakes. While precision is unrealizable, this serves as a more accurate guide to reality than popular science expositions. All chapters can be read independently. The \textbf{essence} is accessible to anyone, mostly in the chapter summaries. Some chapters are extremely mathematical and may not appeal to unfamiliar readers. 

Here we go.



\begin{tcolorbox}[
    colback=red!5,
    colframe=red!60!black,
    boxrule=1pt,
    arc=0.5mm,
    left=10pt,
    right=10pt,
    top=10pt,
    bottom=10pt,
    width=\textwidth,
    sharp corners=south,
    breakable,
    title=\textbf{DRAFT WARNING}
]
\setlength{\parskip}{1em}

This is a \textbf{very early draft}. Parts of it are placeholders. Some claims may be wrong.

\end{tcolorbox}
