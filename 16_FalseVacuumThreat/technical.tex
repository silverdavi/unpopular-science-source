\begin{technical}
{\Large\textbf{False Vacuum Decay: Mathematical Formulation}}\\[0.3em]

\textbf{Higgs Potential and Vacuum Stability}\\[0.5em]
The Higgs potential in the Standard Model takes the form
$$
V(\phi) = \mu^2 \phi^2 + \lambda \phi^4,
$$
where $\phi$ is the Higgs field, $\mu^2 < 0$ for spontaneous symmetry breaking, and $\lambda > 0$ for stability. The vacuum expectation value is $\langle \phi \rangle = v = \sqrt{-\mu^2/\lambda} \approx 246$ GeV.

However, renormalization group running modifies the effective potential at high energies. The quartic coupling evolves as
\begin{align*}
(16\pi^2)\,\beta_\lambda &= 12\lambda^2 + (12 y_t^2 - 9 g^2 - 3 g'^2)\lambda \\
&\quad - 12 y_t^4 + \frac{9}{8} g^4 + \frac{3}{4} g^2 g'^2 + \frac{3}{8} g'^4,
\end{align*}
where $y_t$ is the top quark Yukawa coupling, $g$ and $g'$ are the SU(2)$ _L$ and U(1)$ _Y$ gauge couplings, and $Q$ is the energy scale. For Higgs mass $m_H \approx 125$ GeV and top mass $m_t \approx 173$ GeV, $\lambda$ runs negative at scales $Q \sim 10^{10}$--$10^{11}$ GeV, creating a second minimum at large field values.

\textbf{Coleman-De Luccia Instanton}\\[0.5em]
Vacuum decay proceeds via bubble nucleation described by the Euclidean action
$$
S_E = \int d^4x \left[\frac{1}{2}(\partial_\mu \phi)^2 + V(\phi)\right].
$$
The critical bubble solution has $O(4)$ symmetry in Euclidean space, satisfying
$$
\frac{d^2\phi}{d\rho^2} + \frac{3}{\rho}\frac{d\phi}{d\rho} = \frac{dV}{d\phi},
$$
where $\rho = \sqrt{x_1^2 + x_2^2 + x_3^2 + x_4^2}$ is the four-dimensional radius.

The nucleation rate per unit volume is
$$
\Gamma = A e^{-S_E/\hbar},
$$
where $A$ is a prefactor and $S_E$ is the Euclidean action of the bounce solution. For the Standard Model, current estimates give $S_E/\hbar \sim 400-500$, making spontaneous decay negligible over cosmic timescales.

\textbf{High-Energy Triggers}\\[0.5em]
Local few-particle collisions (in colliders or from ultra-high-energy cosmic rays) are not expected to nucleate the required $O(4)$-symmetric critical bubble. Observed cosmic rays reach energies up to $\sim 3\times 10^{20}$ eV without any indication of catalyzed vacuum decay, consistent with the nonperturbative, extended nature of the tunneling process.

\textbf{Bubble Dynamics}\\[0.5em]
Once nucleated, the bubble wall accelerates due to the pressure difference between vacua. The wall Lorentz factor obeys
$$
\gamma^2 = \frac{1}{1-v^2},
$$
where $v$ is the wall velocity. In the thin-wall limit, the pressure difference $\epsilon$ across the wall drives $v$ rapidly toward the speed of light ($v \to 1$) as the bubble expands; the detailed dynamics depend on the surface tension $\sigma$, the energy difference $\epsilon$, and the background spacetime.

\textbf{Renormalization Group Uncertainty}\\[0.5em]
The stability analysis depends critically on precise measurements of Standard Model parameters. The most sensitive quantities are:
\begin{align*}
m_t &= 173.1 \pm 0.9 \text{ GeV} \\
m_H &= 125.25 \pm 0.17 \text{ GeV} \\
\alpha_s(M_Z) &= 0.1179 \pm 0.0010
\end{align*}
An increase of order $\sim$1 GeV in the top mass would shift the stability assessment toward instability, while a $\sim$3 GeV increase in the Higgs mass would favor absolute stability.

\vspace{0.5em}
\textbf{References:}\\
{\footnotesize
Coleman \& De Luccia, \textit{Phys. Rev. D} \textbf{21}, 3305 (1980).\\
Degrassi et al., \textit{JHEP} \textbf{08}, 098 (2012).
}
\end{technical}
