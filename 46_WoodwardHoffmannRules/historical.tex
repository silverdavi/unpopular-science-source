\begin{historical}
In 1952, Kenichi Fukui introduced the concept of frontier molecular orbitals (FMOs), highlighting their role in determining chemical reactivity. His framework, though qualitative, pointed toward a deeper understanding of how electronic structure governs reaction pathways. Around the same time, pericyclic reactions — concerted transformations like electrocyclizations and sigmatropic shifts — presented puzzling stereospecific behavior that defied classical explanations. These reactions proceeded under thermal or photochemical conditions with outcomes that seemed predictable only in hindsight.

In 1965, Robert Burns Woodward, already acclaimed for his intricate natural product syntheses, partnered with Roald Hoffmann, a theoretical chemist then developing orbital phase methods using the extended Hückel approach. Their collaboration produced a groundbreaking series of papers articulating what would become known as the Woodward–Hoffmann rules. They demonstrated that pericyclic reactions followed strict constraints based on the conservation of molecular orbital symmetry. Whether a reaction was allowed or forbidden could be deduced by analyzing how the symmetries of occupied orbitals evolved along a reaction coordinate.

This theoretical insight enriched organic chemistry. Experimentalists quickly began testing the rules across a wide range of rearrangements — electrocyclic closures, sigmatropic shifts, and cycloadditions — all of which showed outcomes that conformed to the predicted symmetry constraints. The rules offered not just post hoc explanation, but predictive power. By the early 1970s, orbital symmetry had become a central organizing principle in mechanistic organic chemistry.

In 1981, Roald Hoffmann shared the Nobel Prize in Chemistry with Kenichi Fukui for their theoretical contributions to reaction mechanisms. Woodward, who had died in 1979, was ineligible for the prize, despite his central role. Still, the legacy of the collaboration was undeniable: it provided a rigorous bridge between quantum chemistry and synthetic strategy, unifying structure, reactivity, and theory in a way that permanently reshaped the discipline.
\end{historical}
