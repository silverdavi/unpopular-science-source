\begin{historical}
In the mid-19th century, Michael Faraday proposed that ice possesses a thin, liquid-like surface layer even below its melting point — a hypothesis based on observations of regelation and contact phenomena. Around the same time, James Thomson and later Lord Kelvin developed the thermodynamic framework of pressure melting, suggesting that applied pressure lowers the melting point and produces a lubricating film. John Joly applied this idea to ice skating in 1886, arguing that the narrow blade of a skate generates sufficient localized pressure to melt ice beneath it.

In the early 20th century, questions emerged about whether pressure alone could explain ice's slipperiness, especially at low temperatures. In the 1930s and 1950s, Frank P. Bowden and David Tabor introduced frictional heating as an alternative mechanism, showing that sliding motion could generate enough heat to produce melt layers, complementing or supplanting pressure-induced effects.

By the late 20th century, new experimental tools — such as atomic force microscopy, sum-frequency generation spectroscopy, and X-ray scattering — enabled scientists to probe the molecular structure of ice surfaces directly. These studies showed that even in the absence of pressure or friction, the outermost molecular layers of ice are inherently disordered and mobile. Molecular dynamics simulations further supported this view, confirming the existence of a quasi-liquid layer driven by the undercoordination of surface molecules.

Together, these historical developments trace a change from macroscopic mechanical theories to microscopic interfacial physics. Ice’s slipperiness, once attributed solely to melting, is now understood as the result of an intrinsic, dynamic surface layer whose mobility increases with temperature — an insight that unifies over a century of observation, theory, and experimentation.
\end{historical}
