\begin{SideNotePage}{
  \textbf{Quasi-Liquid Layer on Ice:}  
  Ice surfaces are coated with a thin, disordered layer of mobile water molecules — called the \emph{quasi-liquid layer}. Even below freezing, this layer behaves like a liquid: molecules at the surface are less tightly bound than those in the bulk lattice, enabling them to rearrange and flow. This surface mobility reduces friction and is a primary reason ice feels slippery, even without pressure or frictional heating. \par
}{42_IceSlipperiness/42_ Wet, Cold, Slippery Slope.pdf}
\end{SideNotePage}
