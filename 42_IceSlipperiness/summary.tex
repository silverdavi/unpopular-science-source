Ice's exceptional slipperiness results primarily from a quasi-liquid layer (QLL) of disordered water molecules at its surface rather than from commonly assumed mechanisms. While pressure melting and frictional heating contribute under specific conditions, neither explains ice's slickness at rest or across wide temperature ranges. Surface molecules, having fewer hydrogen bonds than those in the interior crystal lattice, form a nanometer-thick disordered layer that functions as a molecular lubricant even well below freezing. Counterintuitively, ice is most slippery around -7°C rather than at 0°C, as the QLL is sufficiently mobile at this temperature while the underlying ice remains hard enough to resist deformation.
