\begin{historical}
The recognition that deterministic systems can exhibit unpredictable behavior marked a change in scientific thought. Before the 20th century, classical mechanics — embodied by Newton’s laws — was largely viewed as a complete and exact framework: given initial conditions, future behavior was presumed computable in principle. 

This view was first challenged by Henri Poincaré in the late 1800s, who, in studying the gravitational three-body problem for King Oscar II’s prize, uncovered dynamical instability and nonintegrability. He found that even simple deterministic equations could produce solutions so sensitive to initial conditions that long-term prediction became practically impossible. 

In the decades that followed, these ideas lay mostly dormant until the rise of computers in the mid-20th century allowed for detailed numerical explorations of nonlinear systems. In 1963, Edward Lorenz demonstrated that a set of three differential equations meant to model atmospheric convection could yield drastically different outcomes from imperceptibly different starting points when he re-ran a simulation with rounded initial conditions. This sensitivity, later termed the “butterfly effect,” became the signature of what we now call chaos. 

Rather than disorder or randomness, chaos refers to the intrinsic unpredictability found in certain deterministic systems. It highlighted the limitations of prediction when geometry of the solution space allows tiny differences to grow exponentially, defying long-term computation even in conceptually simple scenarios.
\end{historical}
