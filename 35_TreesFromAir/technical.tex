\begin{technical}
{\Large\textbf{Carbon Fixation and Mass Accumulation in Trees}}

\vspace{0.3em}
\noindent Trees accumulate mass through atmospheric $\mathrm{CO}_2$ fixation powered by sunlight. This section quantifies the chemical and energetic processes converting gaseous carbon into solid biomass.

\vspace{0.5em}
\noindent\textbf{Light-Driven Reactions}

\vspace{0.2em}
\noindent Photosystems I and II generate ATP and NADPH from light energy (680 nm photons ≈ 176 kJ/mol):
\begin{align*}
2\,\mathrm{H}_2\mathrm{O} 
&+ 2\,\mathrm{NADP}^+ 
+ 3\,\mathrm{ADP} 
+ 3\,\mathrm{P}_i 
+ h\nu \nonumber \\
&\rightarrow 2\,\mathrm{NADPH} 
+ 3\,\mathrm{ATP} 
+ \mathrm{O}_2.
\end{align*}
Quantum requirement: 8–10 photons per $\mathrm{CO}_2$ molecule fixed.

\vspace{0.5em}
\noindent\textbf{Carbon Fixation and Biomass Synthesis}

\vspace{0.2em}
\noindent In the Calvin–Benson cycle, carbon dioxide is enzymatically fixed into triose phosphates using the energy carriers from the light reactions. The overall reaction for one glucose unit is:
\begin{align*}
6\,\mathrm{CO}_2 
&+ 18\,\mathrm{ATP} 
+ 12\,\mathrm{NADPH} \nonumber \\
&\rightarrow \mathrm{C}_6\mathrm{H}_{12}\mathrm{O}_6 
+ 18\,\mathrm{ADP} \nonumber \\
&\quad + 18\,\mathrm{P}_i 
+ 12\,\mathrm{NADP}^+.
\end{align*}
Glucose is polymerized into cellulose by dehydration:
\begin{align*}
n\,\mathrm{C}_6\mathrm{H}_{12}\mathrm{O}_6 
&\rightarrow (\mathrm{C}_6\mathrm{H}_{10}\mathrm{O}_5)_n 
+ n\,\mathrm{H}_2\mathrm{O}.
\end{align*}
These polymers form the primary structure of wood (secondary xylem), alongside lignin and hemicellulose.

\vspace{0.5em}
\noindent\textbf{Oxygen Source Identification via Isotope Labeling}

\vspace{0.2em}
\noindent The $^{18}\mathrm{O}$ labeling experiments by Ruben and Kamen (1941) definitively established oxygen source separation:

\noindent\textit{Water source test:}
\begin{align*}
\mathrm{CO}_2 + \mathrm{H}_2^{18}\mathrm{O} + h\nu 
&\rightarrow [\mathrm{CH}_2\mathrm{O}] + ^{18}\mathrm{O}_2
\end{align*}
\textit{Result:} Heavy oxygen ($^{18}\mathrm{O}$) appeared exclusively in released $\mathrm{O}_2$, not in organic products.
\noindent\textit{$\mathrm{CO}_2$ source test:}
\begin{align*}
\mathrm{C}^{18}\mathrm{O}_2 + \mathrm{H}_2\mathrm{O} + h\nu 
&\rightarrow [\mathrm{CH}_2^{18}\mathrm{O}] + \mathrm{O}_2
\end{align*}

\vspace{0.5em}
\noindent\textbf{Energy Storage Density}

\vspace{0.2em}
\noindent Wood represents highly concentrated solar energy storage: \noindent\textbf{Energy density:} 16–20 MJ/kg (dry wood) \noindent\textbf{Solar capture efficiency:} 1–3\% of incident radiation \noindent\textbf{Mature tree storage:} 50–100 GJ total (accumulated over decades) \noindent\textbf{Photon requirement:} ~8–10 photons per $\mathrm{CO}_2$ molecule fixed

\vspace{0.3em}
\noindent This energy density approaches that of fossil fuels, demonstrating that photosynthesis creates a highly efficient biological battery.

\vspace{0.5em}
\noindent\textbf{Quantitative Mass Accumulation}

\vspace{0.2em}
\noindent For annual NPP of $10^4\,\mathrm{kg/ha}$ dry biomass (50\% carbon):
\begin{align*}
\text{$\mathrm{CO}_2$ fixed} &= 18.4\,\mathrm{tonnes}\,\mathrm{CO}_2/\mathrm{ha}/\mathrm{year} \\
\text{Per tree (100/ha)} &= 184\,\mathrm{kg}\,\mathrm{CO}_2/\mathrm{year}
\end{align*}
Over 50 years, each tree accumulates ~2.5 tonnes carbon, corresponding to ~5 tonnes total dry biomass — consistent with mature forest measurements.

\vspace{0.5em}
\noindent\textbf{Elemental Mass Contribution}

\vspace{0.2em}
\noindent Typical dry mass composition:

\noindent Carbon: 45–50\% (from atmospheric $\mathrm{CO}_2$)

\noindent Oxygen: 40–45\% (primarily from $\mathrm{CO}_2$)

\noindent Hydrogen: ~6\% (from water)

\noindent Minerals: 1–5\% (from soil: N, P, K, Ca, etc.)

\vspace{0.5em}
\noindent\textbf{References:}\\
{\footnotesize
Farquhar, G. D., von Caemmerer, S., Berry, J. A. (1980). A biochemical model of photosynthetic $\mathrm{CO}_2$ assimilation in C$_3$ leaves. \textit{Planta}, \textbf{149}, 78–90.\\
Taiz, L., Zeiger, E. (2010). \textit{Plant Physiology}.
}
\end{technical}
