The Envelope Paradox presents two envelopes where one contains twice the money of the other. After selecting one envelope, seemingly valid probabilistic reasoning suggests an expected gain by switching (averaging x/2 with 2x), regardless of which envelope was initially chosen. This symmetric conclusion creates a logical inconsistency since perpetual switching cannot be optimal. The paradox arises from improper application of expected value calculations to scenarios with unbounded distributions or when conditional probabilities are not properly accounted for. Resolving the paradox requires distinguishing between known values and variables, recognizing when probability distributions are ill-defined, and understanding the limitations of calculations with potentially infinite quantities.
