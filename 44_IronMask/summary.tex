The prisoner known as "Eustache Dauger" remained in state custody for thirty-four years (1669-1703) under extraordinary protocols of secrecy. His confinement spanned four locations under the continuous supervision of a single jailer, Bénigne Dauvergne de Saint-Mars. Official correspondence reveals exceptional measures: a specially constructed cell with sound isolation, strict limitations on communication, and a requirement to wear a black velvet mask when visible to anyone outside Saint-Mars's control. The prisoner served as valet to another detainee at Pignerol before eventual transfer to the Bastille, where he died and was buried under the alias "Marchioly."