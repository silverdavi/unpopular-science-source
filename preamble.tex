% Basic document setup
% 7"×10" format with appropriate margins
\usepackage[paperwidth=7in, paperheight=10in, margin=0.75in, inner=0.875in, outer=0.625in]{geometry}
\usepackage[latin]{babel}
\usepackage{graphicx}
\usepackage{booktabs}
\usepackage{epstopdf}
\usepackage{float}
\usepackage{tikz}
\usepackage{tikzsymbols}
\usepackage{pgfplots}
\usepackage{pgffor}
\usepackage{chemfig}
\usepackage{enumitem}
\usepackage{tabularx}
\usepackage{graphicx}
\usepackage{changepage}
\usepackage{verbatim}
\usepackage{comment}
\usepackage{setspace}
\usepackage{ragged2e}
\usepackage{array}
\usepackage{pgffor} % Required for \foreach
\usepackage{ifoddpage}
\usepackage{catchfile} 
\usepackage{xstring}
\usepackage[most]{tcolorbox}  % Ensure this is included in your preamble
\usepackage{array}
\usepackage{colortbl}
\usepackage{caption}
\usepackage{varwidth}
\usepackage{pagecolor}
\usepackage{lipsum} % optional for filler text
\usepackage{eso-pic} % For background picture in sidenotes



\pgfplotsset{compat=1.18}

% Typography and text formatting
\usepackage[protrusion=true,expansion=true]{microtype}
\usepackage{CJK}
\newcommand{\katakana}[1]{\begin{CJK}{UTF8}{min}#1\end{CJK}}
\usepackage{shadowtext}
\usepackage{bbding}
\usepackage{textcomp}
\usepackage{hyperref}
\usepackage{parskip}
\usepackage{multicol}
\usepackage{needspace}
\usepackage{etoolbox}
\usepackage{pifont}  % for nice symbols like the filled circle

% Math packages
\usepackage{amsmath, amssymb, physics}
\usepackage{unicode-math}

% Multilingual support
\usepackage{polyglossia}
\usepackage{amsfonts}
\usepackage{lmodern} 
\setdefaultlanguage{english}
\setotherlanguage{greek}
\setotherlanguage{hebrew}
\setotherlanguage{sanskrit}
\setotherlanguage{arabic}

% Assign fonts to polyglossia languages
\newfontfamily\greekfont[Script=Greek]{Linux Libertine O}
\newfontfamily\hebrewfont[Script=Hebrew]{Arial Hebrew}
\newfontfamily\sanskritfont[Script=Devanagari]{Devanagari Sangam MN}
\newfontfamily\arabicfont[Script=Arabic,Scale=1.2]{DecoType Naskh}

% Font configuration
\usepackage{fontspec}
\setmainfont{Libertinus Serif}                    % Main serif font
\newfontfamily\historyfont{Crimson Pro}           % Historical section font
          % Technical section font
\newfontfamily\commentaryfont{Crimson Pro}        % Commentary font


\newfontfamily\piefontfamily{Charis SIL}
\newfontfamily\ipafontfamily{Charis SIL}

% Create properly scoped commands that preserve current font size
\newcommand{\piefont}[1]{{\piefontfamily #1}}
\newcommand{\ipafont}[1]{{\ipafontfamily #1}}

\newfontfamily\summaryfont{Libertinus Serif Italic} % Summary font
\newfontfamily\technicalfont{Libertinus Serif}

% Color definitions
\usepackage{xcolor}
\definecolor{historycolor}{RGB}{70,30,0}         % Warm brown for historical sections
\definecolor{technicalcolor}{RGB}{0,20,60}       % Deep blue for technical sections
\definecolor{commentarycolor}{RGB}{0,0,0}        % Black for commentary
\definecolor{summarycolor}{RGB}{90,90,90}        % Gray for summaries
\definecolor{linkcolor}{RGB}{0,85,155}           % Link color for hyperlinks
\definecolor{SUMMARYCOLOR}{RGB}{90,90,90}        % Gray for summaries (uppercase)
\definecolor{lavender}{RGB}{230,230,250}
\definecolor{lightgray}{gray}{0.85}


% Section styling with titlesec
\usepackage{titlesec}
\usepackage{tocloft}  % For customizing TOC

\titleformat{\chapter}[hang]
    {\normalfont\huge\bfseries}{\thechapter}{1em}{} % Chapter title styling
\titlespacing*{\chapter}{0pt}{0pt}{10pt}            % Adjust spacing around chapters
\titleformat{\section}[hang]
    {\normalfont\Large\bfseries}{\thesection}{1em}{} % Section title styling
\titlespacing*{\section}{0pt}{10pt}{5pt}             % Adjust spacing around sections

% Frame and box packages
\usepackage[framemethod=TikZ,skipabove=6pt,skipbelow=6pt]{mdframed}
\usetikzlibrary{decorations.pathmorphing, decorations.shapes, decorations.footprints, shapes.geometric, positioning, patterns, fit,arrows.meta, decorations.pathmorphing, backgrounds, calc, decorations.fractals}
\mdfsetup{splitbottomskip=2pt, splittopskip=2pt}

\usetikzlibrary{lindenmayersystems}

% Spacing and layout settings
\setlength{\parindent}{0pt}      % No paragraph indentation
\setlength{\parskip}{0.5em}      % Paragraph spacing for better readability
\setlength{\columnsep}{15pt}     % Adjust column separation for multicol layout


% Force content to start on the next even-numbered page

\newcommand{\startchapter}{%
  \clearpage
  \checkoddpage
  \ifoddpage
    % already odd — do nothing
  \else
    \hbox{}
    \thispagestyle{empty}
    \clearpage
  \fi
}




% TOC customization for chapter summaries
\setlength{\cftbeforechapskip}{1.0em}  % Increase space between TOC entries
\setlength{\cftchapindent}{0em}        % Chapter indent in TOC
\renewcommand{\cftchapfont}{\bfseries} % Chapter font in TOC

% Header setup with fancyhdr - MUST be loaded after titlesec
\usepackage{fancyhdr}
\setlength{\headheight}{14pt}  % Slightly more than the required 13.59999pt

% Create chapter title extraction command to handle only the title part
\makeatletter
\def\extracttitle#1\\#2\@nil{#1}
\makeatother

% Define the main page style
\pagestyle{fancy}
\fancyhf{} % Clear all header and footer fields
\fancyhead[LE,RO]{\thepage}
\fancyhead[RE]{\textit{\leftmark}}
\fancyhead[LO]{\textit{\rightmark}}
\renewcommand{\headrulewidth}{0.4pt}
\renewcommand{\footrulewidth}{0pt}

% Fix the chapter page style (first page of each chapter)
\fancypagestyle{plain}{%
  \fancyhf{} % Clear all header and footer fields
  \fancyfoot[C]{\thepage} % Just page number at bottom
  \renewcommand{\headrulewidth}{0pt} % No header rule on first page
}

% Modify chaptermark to extract just the title
% Better approach to handle headers without the summary
\makeatletter
% Simple approach - standard chapter marking
\renewcommand{\chaptermark}[1]{%
  \markboth{\MakeUppercase{\chaptername\ \thechapter.\ #1}}{}}
\renewcommand{\sectionmark}[1]{\markright{\thesection.\ #1}}

% Helpers to input title and summary from file
\newcommand{\inputtitle}[1]{\IfFileExists{#1/title.tex}{A Circle of Time
}{MissingTitle}}
\newcommand{\inputsummary}[1]{\IfFileExists{#1/summary.tex}{The yellow color of gold requires relativistic quantum mechanics to explain, unlike silver's silvery appearance. Electrons in gold atoms reach 58\% of light speed, causing changes in the 6s and 5d orbitals. This shifts absorption to blue wavelengths, resulting in the reflection of yellow-red light. Similar relativistic effects explain mercury's liquid state and platinum's white appearance. These everyday properties demonstrate how modern physics manifests in macroscopic observations.
}{MissingSummary}}

\newtcolorbox{shadedstory}[1][]{
  colback=gray!10,
  colframe=gray!50,
  sharp corners,
  boxrule=0.8pt,
  left=10pt,
  right=10pt,
  top=10pt,
  bottom=10pt,
  title=#1,
  fonttitle=\bfseries\large,
  enhanced,
}



\newcommand{\chapterseparator}{%
  \begin{center}
    % Include chapter-specific fractal tree PNG from fractal_trees/with_fruits/
    \includegraphics[height=3.5cm]{fractal_trees/with_fruits/\thechapter.png}
  \end{center}
}





% Custom chapter with summary command (TOC only, no visible title)
\newcommand{\chapterwithsummary}[3][]{%
  % #1 = optional label
  % #2 = chapter title
  % #3 = chapter summary

  % TOC entry only (multi-line)
  \addcontentsline{toc}{chapter}{%
    \protect\numberline{\thechapter}#2\\
    {\normalfont\small\textit{\textcolor{summarycolor}{#3}}}%
  }

  % Update header with title only (no page title)
  \chaptermark{#2}%

  % Apply label if provided
  \ifx\\#1\\\else\label{#1}\fi

  % Avoid rendering a visible chapter heading
  \phantomsection % for correct link targets
  \vspace*{-3em} % suppress visual whitespace
}


\newread\file
\newcommand{\readfirstline}[2]{%
  \openin\file=#2
  \read\file to \temp
  \closein\file
  \xdef#1{\temp}%
  % Ensure temp is not accidentally output
}
\newcommand{\chapterwithsummaryfromfile}[2][]{%
  \refstepcounter{chapter}
  \IfFileExists{#2/title.tex}{
    \readfirstline{\chaptertitle}{#2/title.tex}
  }{
    \def\chaptertitle{Missing Title}
  }
  \IfFileExists{#2/summary.tex}{
    \readfirstline{\chaptersummary}{#2/summary.tex}
  }{
    \def\chaptersummary{No summary available.}
  }
  
  % Store title and summary for later TOC entry and header (removed immediate TOC entry)
  \gdef\storedchaptertitle{\chaptertitle}
  \gdef\storedchaptersummary{\chaptersummary}
  
  % Do NOT set chaptermark here - delay until actual chapter content begins
  \ifx\\#1\\\else\label{#1}\fi
  \phantomsection
  \vspace*{-3em}
}

\makeatother

% Fix hyperref issues with multi-line TOC entries
\makeatletter
\patchcmd{\@chapter}
  {\addcontentsline{toc}{chapter}%
    {\protect\numberline{\thechapter}#1}}
  {\addcontentsline{toc}{chapter}%
    {\protect\numberline{\thechapter}#1}}
  {}{}
\makeatother

% Hyperlink settings - must be AFTER all other settings
\usepackage{hyperref}
\pdfstringdefDisableCommands{%
  \def\\{ }% Replace \\ with space in PDF strings
}
\hypersetup{
    colorlinks=true,
    linkcolor=linkcolor,
    citecolor=linkcolor,
    urlcolor=linkcolor
}

% Custom environments
% Historical section styling
\newenvironment{historical}{%
    \begin{mdframed}[%
        linecolor=historycolor, 
        linewidth=0.5pt, 
        backgroundcolor=white, 
        topline=false, 
        bottomline=false, 
        rightline=false, 
        leftmargin=1em, 
        rightmargin=1em]
    \historyfont\color{historycolor}
}{%
    \end{mdframed}
}

% Technical environment with improved multicols handling
\newenvironment{technical}{%
    \par\medskip\noindent
    \begin{tcolorbox}[
        enhanced,
        colframe=technicalcolor,
        colback=gray!5,
        boxrule=0.8pt,
        arc=0mm,
        left=5pt,
        right=5pt,
        top=5pt,
        bottom=10pt,
        breakable=false,
        parbox=false,
        after={\par\nobreak\noindent}  % Prevent page break after the box
    ]
    \begingroup % Begin a local group for scoping
    \renewcommand{\bfseries}{\fontseries{b}\selectfont} % Ensure bold works
    \renewcommand{\itshape}{\fontshape{it}\selectfont}  % Ensure italics works
    \renewcommand{\emph}[1]{\textit{##1}}               % Escape # properly
    \technicalfont\small % Apply technical font and small size
    \sloppy % Allow more flexible line breaking
    \hyphenpenalty=50 % Make hyphenation more likely
    \exhyphenpenalty=50 % Make hyphenation of already hyphenated words more likely
    
    % Adjust itemize spacing for technical environment
    \setlist[itemize]{nosep, leftmargin=1em, itemsep=0.1em, topsep=0.3em, partopsep=0pt}
    
    % Use multicols directly with adjusted parameters
    \begin{multicols}{2}
    \setlength{\columnseprule}{0.1pt} % Thin rule between columns
    \setlength{\columnsep}{12pt} % Slightly reduced column separation
}{%
    \end{multicols}
    \endgroup % End the local group, restoring global settings
    \end{tcolorbox}
}

% Commentary environment styling
\newenvironment{commentary}[1][Commentary]{%
    \bigskip % Add vertical space before commentary
    % Title with appropriate styling
    \begin{center}
        {\large\itshape\bfseries\textcolor{commentarycolor}{#1}}
    \end{center}
    \vspace{0.5em}
    % Content with right border
    \begin{mdframed}[%
        linecolor=commentarycolor, 
        linewidth=1pt, 
        backgroundcolor=white, 
        topline=false, 
        bottomline=false, 
        rightline=true, 
        leftmargin=1em, 
        rightmargin=1em]
    \commentaryfont\color{commentarycolor}
    \RaggedRight
}{%
    \end{mdframed}
}

\newcommand{\centeredfullpage}[2]{%
  \clearpage
  \thispagestyle{empty}

  \vspace*{\fill}

  \begin{center}
    \begin{minipage}{0.8\textwidth}
      \centering
      #1

      \vspace{1em}

      {\fontsize{12pt}{14pt}\selectfont \textbf{\textit{#2}}}
    \end{minipage}
  \end{center}

  \vspace*{\fill}

}



% Humor section for jokes and funny stories
\newenvironment{humorbox}[1][Lighter Side]{%
    \begin{figure}[b]  % Position at bottom of page by default
        \begin{tcolorbox}[
            enhanced,
            colframe=gray!50!black,
            colback=gray!5,
            boxrule=0.5pt,
            arc=2mm,
            left=10pt,
            right=10pt,
            top=6pt,
            bottom=6pt,
            title=#1,
            fonttitle=\bfseries\itshape,
            coltitle=black,
            breakable,
            parbox=false
        ]
        \itshape  % Make the text italic
}{%
        \end{tcolorbox}
    \end{figure}
}

\newcommand{\fullpagehumor}[2][Lighter Side]{%
  \clearpage
  \thispagestyle{empty}
  
  \vspace*{\fill}
  \noindent
  \begin{tcolorbox}[
      enhanced,
      width=\textwidth,
      colframe=gray!50!black,
      colback=gray!5,
      colbacktitle=gray!30,
      coltitle=black,
      boxrule=0.5pt,
      arc=2mm,
      left=15pt,
      right=15pt,
      top=10pt,
      bottom=10pt,
      title=#1,
      fonttitle=\bfseries\itshape
  ]
    \itshape #2
  \end{tcolorbox}
  \vspace*{\fill}
}



\newenvironment{exercisebox}[1][Exercises]{%
    \begin{tcolorbox}[
        enhanced,
        breakable,  % <-- Add this
        colframe=blue!60!black,
        colback=blue!5,
        boxrule=0.5pt,
        arc=2mm,
        left=15pt,
        right=15pt,
        top=5pt,
        bottom=10pt,
        title=#1,
        fonttitle=\bfseries,
        coltitle=white,
        colbacktitle=blue!60!black,
        attach boxed title to top left={yshift=-2mm, xshift=5mm},
        boxed title style={arc=1mm, boxrule=0.5pt}
    ]
    \begin{enumerate}[leftmargin=*]
}{%
    \end{enumerate}
    \end{tcolorbox}
}


% Full-page exercises section
\newcommand{\fullpageexercises}[2][Exercises]{%
  \clearpage                  % start it on a fresh page
  \thispagestyle{empty}       % no header/footer on page one of the box
  \begin{center}
    \begin{tcolorbox}[
      enhanced,
      breakable,               % < — allow the box to span pages
      width=\textwidth,
      colframe=blue!60!black,
      colback=blue!5,
      colbacktitle=blue!60!black,
      coltitle=white,
      boxrule=0.5pt,
      arc=2mm,
      left=5pt,
      right=5pt,
      top=10pt,
      bottom=10pt,
      title=#1,
      fonttitle=\bfseries,
      attach boxed title to top left={yshift=-2mm, xshift=5mm},
      boxed title style={arc=1mm, boxrule=0.5pt}
    ]
      #2
    \end{tcolorbox}
  \end{center}
  % **no** \clearpage or \cleardoubleoddpage here
}



\newcommand{\imagefigure}[2]{%
  \thispagestyle{empty}
  \begin{center}
    \vspace*{2em} % Small spacing instead of \null\vfill
    #1

    \vspace{1em}

    % Caption (only if nonempty)
    % Adjusted margins for 7"×10" format (was 2cm each side)
    \begin{adjustwidth}{1.5cm}{1.5cm}
      {\fontsize{11pt}{14pt}\selectfont #2}
    \end{adjustwidth}

  \end{center}
  \vspace{2em}
}


% Image with caption vertically centered in remaining space
% Usage: \inlineimage{width_factor}{path/to/image.png}{Caption text}
% Example: \inlineimage{0.8}{image.png}{My caption} uses 0.8\linewidth
\newcommand{\inlineimage}[3]{%
  \par
  \vspace{0pt plus 1fill}% Flexible space above for vertical centering
  \begin{center}
    \includegraphics[width=#1\linewidth,keepaspectratio]{#2}\\[0.5em]
    {\small\textit{#3}}
  \end{center}
  \vspace{0pt plus 1fill}% Flexible space below for vertical centering
}


% Topic map macro: render as small caps with ○ separator
\newcommand{\topicmap}[1]{%
  \noindent\textsc{%
    \def\tempa{}%
    \foreach \x in {#1} {%
      \ifx\tempa\empty
        \x
      \else
        \enspace$\circ$\enspace\x
      \fi
      \xdef\tempa{nonempty}%
    }%
  }%
}




\newcommand{\storyintro}[1]{%
  \clearpage
  \thispagestyle{empty}
  \begin{center}
    \vspace*{\fill}

    % Title
    {\Huge \bfseries A Circle of Time
}

    \vspace{1em}

    % Summary
    % Summary (cleaner, no forced italics)
    \vspace{1em}
    \begin{minipage}{0.75\textwidth}
        \centering
        {\Large \color{summarycolor} The yellow color of gold requires relativistic quantum mechanics to explain, unlike silver's silvery appearance. Electrons in gold atoms reach 58\% of light speed, causing changes in the 6s and 5d orbitals. This shifts absorption to blue wavelengths, resulting in the reflection of yellow-red light. Similar relativistic effects explain mercury's liquid state and platinum's white appearance. These everyday properties demonstrate how modern physics manifests in macroscopic observations.
}
    \end{minipage}


    \vspace{1.5em}

    % Separator
    {\Large \textcolor{gray}{\ding{108}}}

    \vspace{1em}

    % Topicmap
    {\normalsize \textsc{\topicmap{
Solar Core Fusion,
Proton-Proton Chain,
Coulomb Barrier Problem,
Quantum Tunneling Solution,
Gamow Factor,
Weak Force \& Neutrinos,
Lepton Number Conservation,
Solar Neutrino Problem,
Neutrino Oscillations,
Hydrostatic Equilibrium,
Main Sequence Stability
}}}

    \vspace*{\fill}
  \end{center}
  \clearpage
}





\newcommand{\inputstory}[1]{%
    % === CHAPTER STRUCTURE: EXACTLY 10 PAGES ===
    % 1. Big Title (recto) - 1 page
    % 2. Sidenote (verso) - 1 page  
    % 3. Title+Summary+Topicmap+Quote (recto) - 1 page
    % 4-8. Historical+Main content - 5 pages
    % 9. Technical - 1 page
    % 10. Empty verso page - 1 page (for chapter separation)
    %
    % VERSO EMPTY PAGE BEFORE EACH CHAPTER (including first)
    % This ensures proper recto/verso alignment and 10-page structure
    % Every chapter gets a verso separator page for consistent alignment
    \clearpage
    \thispagestyle{empty}
    \mbox{}
    \clearpage
    
    % Start chapter on new page - chapter numbering handled by \chapterwithsummaryfromfile
    \clearpage

    % --- PAGE 1: Dedicated Title Page (big, centered) ---
    \phantomsection
    \readfirstline{\chaptertitle}{#1/title.tex}
    \readfirstline{\chaptersummary}{#1/summary.tex}

    \thispagestyle{empty}
    \begin{center}
        \vspace*{\fill}
        {\fontsize{48pt}{62pt}\selectfont\bfseries\raggedright
        \parbox{0.8\textwidth}{\centeringA Circle of Time
}}
        \vspace*{\fill}
    \end{center}

    \clearpage

    % --- PAGE 2: Sidenote ---
    \IfFileExists{#1/sidenote.tex}{%
        \begin{SideNotePage}{
  \textbf{Top (Dimensional Scaling of Inverse Laws):}  
  The same point-source strength spreads differently depending on spatial dimension. In 1D, the influence remains constant along a line. In 2D, the effect dilutes like $1/r$ as it spreads over a circle. In 3D, it falls off as $1/r^2$, spreading over the surface of a sphere. This explains why gravitational and electrostatic forces scale as inverse-square laws in 3D space. \par

  \textbf{Bottom (Gabriel’s Horn and the Painter’s Paradox):}  
  A surface of revolution formed by rotating $y = 1/x$ around the $x$-axis for $x > 1$. Though the horn extends infinitely, it encloses a finite volume ($\int_1^\infty \pi (1/x)^2 dx = \pi$) but has infinite surface area ($2\pi \int_1^\infty (1/x) \sqrt{1 + (1/x)^2} dx = \infty$). Paradoxically, one could “fill” it with a finite amount of paint, but never coat its inner surface. \par

}{24_FourDSpacetime/24_ Put on Your 4D Glasses.pdf}
\end{SideNotePage}
%
    }{%
        % If no sidenote, create empty page
        \thispagestyle{empty}
        \mbox{}
    }{}

    \clearpage

    % Add TOC entry HERE (after sidenote page, to show correct page in PDF TOC)
    \addcontentsline{toc}{chapter}{%
      \protect\numberline{\thechapter}\storedchaptertitle\\
      {\normalfont\small\textit{\textcolor{summarycolor}{\storedchaptersummary}}}%
    }

    \clearpage

    % --- PAGE 3: Title + Summary + Topicmap + Quote (original intro style) ---
    \thispagestyle{empty}
    \begin{center}
        \vspace*{\fill}

        {\Huge \bfseries A Circle of Time
}

        \vspace{2em}
        \begin{minipage}{0.8\textwidth}
            {\fontsize{13pt}{18pt}\selectfont\color{black}
            \justifying
            The yellow color of gold requires relativistic quantum mechanics to explain, unlike silver's silvery appearance. Electrons in gold atoms reach 58\% of light speed, causing changes in the 6s and 5d orbitals. This shifts absorption to blue wavelengths, resulting in the reflection of yellow-red light. Similar relativistic effects explain mercury's liquid state and platinum's white appearance. These everyday properties demonstrate how modern physics manifests in macroscopic observations.
}
        \end{minipage}

        \vspace{2em}

        \chapterseparator

        \vspace{2em}

        \IfFileExists{#1/topicmap.tex}{%
            \begin{minipage}{0.7\textwidth}
                \centering
                 \topicmap{
Solar Core Fusion,
Proton-Proton Chain,
Coulomb Barrier Problem,
Quantum Tunneling Solution,
Gamow Factor,
Weak Force \& Neutrinos,
Lepton Number Conservation,
Solar Neutrino Problem,
Neutrino Oscillations,
Hydrostatic Equilibrium,
Main Sequence Stability
}
            \end{minipage}
        }{}

        \vfill

        \IfFileExists{#1/quote.tex}{%
            \vspace{2em}
            \begin{minipage}{0.8\textwidth}
                \centering \itshape 
\begin{flushright}
\emph{"Das Unendliche hat wie keine andere Frage von jeher so tief das Gemüt der Menschen bewegt... Aus dem Paradies, das Cantor uns geschaffen, soll uns niemand vertreiben können."} \\
("The infinite! No other question has ever moved so profoundly the spirit of man... Cantor's paradise, from which no one will expel us.") \\
— David Hilbert, 1926
\end{flushright}

            \end{minipage}
        }{}

        \vspace*{\fill}
    \end{center}

    \clearpage

    % --- PAGES 4-8: Historical + Main + Optional materials (exactly 5 pages total) ---
    % Set chapter header HERE when content actually begins
    \chaptermark{\storedchaptertitle}
    {\LARGE \bfseries A Circle of Time
}
    
\begin{historical}
In 1959, Paul V. C. Hough filed a patent — issued in 1962 as U.S. Patent 3,069,654 — for a method of identifying complex patterns in visual data by mapping image features into a parameter space. His technique framed detection as finding parameter settings supported by many edge points. The familiar sinusoidal loci in a polar-coordinate domain arise with the normal parameterization \((\rho,\theta)\), which was formalized and popularized subsequently.

Early implementations relied on analog hardware and optical computing elements. Limitations in memory and processing speed constrained accumulator resolution and the range of detectable geometries. Despite these constraints, the method was adopted in early automation systems for detecting weld lines, highway markings, and parts in mechanical assemblies.

In 1972, Richard Duda and Peter Hart provided the first formal exposition of the method in their landmark paper, reframing it as a discrete voting process over a bounded parameter space and introducing the normal parameterization \(\rho = x\cos\theta + y\sin\theta\) that treats all orientations uniformly. Their formulation gave the method the name "Hough transform" and connected it to broader principles in statistical decision theory. Subsequent extensions by Ballard and others generalized the idea to arbitrary curves and spatial templates, enabling detection of circles, ellipses, and parabolas.

By the 1980s and 1990s, with advances in digital signal processors and parallel computing, the Hough transform became a foundational tool in industrial vision systems, robotics, and medical image reconstruction. Its structure-preserving mapping from image to parameter space allowed for robust detection even in the presence of occlusion, fragmentation, and noise  —  a capacity that remains central to modern implementations in lane detection, tomography, and motion analysis. 
\end{historical}
    The Sun produces energy through nuclear fusion in its core. Gravitational compression generates densities exceeding $150\,\text{g}/\text{cm}^3$ and temperatures near $1.5 \times 10^7\,\text{K}$. At these conditions, hydrogen nuclei fuse into helium, releasing binding energy.

In the simplest view, fusion proceeds via close approaches of hydrogen nuclei aided by quantum tunneling. At core temperatures of order $10^7\,\text{K}$, protons have thermal kinetic energies too small to classically overcome the Coulomb barrier, but tunneling allows occasional close encounters where the strong nuclear force binds them. Through a sequence of interactions known as the proton–proton chain, four protons are ultimately transformed into a helium nucleus. 

Each fusion event in the Sun's core releases a small amount of energy: approximately $26.7\,\text{MeV}$ per helium nucleus formed. However, the Sun generates a total power output of roughly $3.8 \times 10^{26}\,\text{W}$, which requires converting mass to energy at an enormous rate. By the relation $E = mc^2$, this luminosity implies a mass loss of about $4.3 \times 10^9\,\text{kg}$ per second.

This mass loss manifests as outward radiation pressure. Within the core, energy liberated by fusion builds up pressure that counteracts gravitational collapse. The resulting hydrostatic equilibrium maintains the Sun's structure — every second, the immense weight of the Sun's outer layers is balanced by pressure generated from fusing approximately $6 \times 10^{11}$ kilograms of hydrogen into helium. The Sun's long-term stability emerges from this balance. Fusion sustains the outward force needed to resist the crushing pull of its own mass.

The energy generated in the core undergoes radiative diffusion. Photons scatter innumerable times off electrons and nuclei as they migrate outward through the radiative zone. In the outer layers, convective transport becomes dominant, with rising and sinking plasma transporting energy. After this migration, energy is finally emitted from the photosphere as sunlight, spanning a broad electromagnetic spectrum.

Conservation of energy, momentum, and electric charge ensures consistency in nuclear reactions. Quantum field theories of particle interactions also impose another conserved quantity: lepton number. Leptons — a class of particles including electrons, neutrinos, and their antiparticles — must be created or destroyed in such a way that the total lepton number remains unchanged.

The proton–proton chain, which powers the Sun, involves changes in particle types that require mechanisms beyond the electromagnetic and strong forces. In particular, the weak nuclear force is necessary to enable the conversion of protons into neutrons while preserving all conservation laws. The weak force enables the fusion of hydrogen into helium.

Here is the first step of the chain:
\[
\text{p} + \text{p} \;\to\; \text{d} + e^+ + \nu_e,
\]

where $\text{p}$ denotes a proton, $\text{d}$ a deuteron (a bound state of one proton and one neutron), $e^+$ a positron, and $\nu_e$ an electron neutrino. In this reaction, one proton transforms into a neutron through a weak interaction. To conserve electric charge, a positron — the antimatter counterpart of the electron — is emitted. To conserve lepton number, an electron neutrino is emitted simultaneously. 

In the lepton number accounting, electrons and neutrinos are assigned a lepton number of $+1$, while positrons and antineutrinos carry a lepton number of $-1$. Before the reaction, the system has zero net lepton number; after the reaction, the positron ($-1$) and neutrino ($+1$) balance each other, maintaining overall neutrality. The emission of the neutrino is therefore a necessity for the reaction to be consistent with the symmetries of particle physics.

Although neutrinos possess extremely small mass and interact only via the weak force, they carry away a significant fraction of the reaction's energy and linear momentum. Unlike photons — which scatter thousands of times before reaching the solar surface — neutrinos traverse the Sun's dense interior with minimal interaction and escape into space almost immediately. Neutrinos produced in the Sun's core reach Earth in about 8 minutes, providing a direct and real-time probe of nuclear processes inside the Sun.

The detection of solar neutrinos has been crucial for confirming theoretical models of stellar energy generation. Measurements not only validate the dominance of the proton–proton chain but also reveal minor contributions from alternative fusion pathways, such as the carbon–nitrogen–oxygen (CNO) cycle in which carbon, nitrogen, and oxygen nuclei fuse to produce helium.

Quantum mechanics introduces behaviors absent in classical physics. One of these is tunneling: the ability of a particle to penetrate and traverse a potential barrier even when its total energy is insufficient to overcome it. 

Classically, a particle with energy less than the height of a potential barrier would be fully reflected, with zero probability of passage. In quantum mechanics, however, particles are described by continuous wavefunctions governed by the Schrödinger equation. Even in classically forbidden regions, the wavefunction persists, decaying exponentially rather than vanishing abruptly.

When a quantum particle encounters a barrier higher than its energy, its wavefunction inside the barrier takes the form of a decaying exponential. If the barrier has finite width, there exists a nonzero probability that the particle will appear beyond the barrier — a phenomenon known as quantum tunneling.

In the solar core, the fusion of protons faces a major obstacle: the Coulomb barrier arising from electrostatic repulsion when the protons are close enough to trigger the strong nuclear force. The potential energy associated with two protons at close approach is approximately $1\,\text{MeV}$, whereas the typical thermal kinetic energy at $1.5 \times 10^7\,\text{K}$ is about $1\,\text{keV}$. Classically, the probability of overcoming the barrier would be vanishingly small, and fusion would be effectively impossible.

Despite this, fusion proceeds because quantum tunneling allows protons to penetrate the Coulomb barrier with nonzero probability. Quantum mechanics enables fusion at energies far below the classical threshold. The proton wavefunctions extend into and through the classically forbidden region, resulting in occasional barrier penetration and subsequent nuclear fusion.

The probability of tunneling through the Coulomb barrier is quantified by the Gamow factor. This factor arises from solving the Schrödinger equation for two charged particles and introduces an exponential suppression depending on the product of the charges, the reduced mass of the system, and the relative kinetic energy. A common parametrization is
\[
P(E) \sim \exp\!\left( -\sqrt{\frac{E_G}{E}} \right),
\]
where $E_G$ (the Gamow energy) depends on the charges and reduced mass. Equivalently, $P(E) \sim \exp(-a/\sqrt{E})$ with a constant $a$ set by the same parameters.

At stellar core temperatures, the Gamow factor dominates the fusion reaction rate. Although tunneling remains rare per collision, the immense number of protons ensures sufficient fusion events to sustain the Sun's energy output. The exponential sensitivity of tunneling probability to temperature creates a self-regulating system: if fusion falls below the rate needed to balance gravitational compression, contraction increases core temperature until equilibrium restores; if fusion runs too high, expansion cools the core and reduces the reaction rate. This feedback mechanism maintains stable stellar burning within a narrow band of core conditions.

This regulatory mechanism underlies the main sequence which is the phase during which hydrogen fusion occurs steadily in the core. A star remains on the main sequence while hydrogen supply sustains the equilibrium fusion rate. The phase lifetime depends on stellar mass, which sets both compression rate and required temperature. For the Sun, this balance produces stability lasting approximately $10^{10}$ years.

The Sun's luminosity remains constant through stable interaction between gravity, fusion kinetics, and quantum tunneling probabilities. These parameters determine the mass-to-energy conversion rate. The resulting energy supports overlying layers without expansion or collapse.

Solar neutrinos arise when the weak force converts a proton's up quark into a down quark during fusion. Baryon number is conserved (two initial protons become a deuteron with baryon number 2), and lepton number is conserved because the emitted positron ($L=-1$) and electron neutrino ($L=+1$) balance to zero.

The Sun produces approximately $2 \times 10^{38}$ neutrinos per second, carrying $2\%$ of fusion energy. With interaction cross-sections of $10^{-44}\,\text{cm}^2$, they pass through matter nearly unimpeded — while photons require thousands of years to diffuse through the Sun, neutrinos escape instantaneously, reaching Earth in 8 minutes.

Every detected neutrino was produced moments earlier in the solar core. Measuring their flux and energy spectrum tests stellar energy generation models with high precision.

When physicists first detected solar neutrinos in the 1960s, they encountered a puzzle. Raymond Davis Jr.'s Homestake experiment used 400,000 liters of perchloroethylene to capture neutrinos through the reaction:
\[
\nu_e + \text{Cl}^{37} \;\to\; e^- + \text{Ar}^{37}.
\]

The Homestake detector measured only about one-third of the neutrino flux predicted by standard solar models. This deficit, known as the solar neutrino problem, persisted for over three decades despite improved experiments and refinements to stellar theory.

The resolution came through discovering neutrino oscillations — neutrinos transform between different flavors as they propagate. The Standard Model lists three flavors: electron ($\nu_e$), muon ($\nu_\mu$), and tau ($\nu_\tau$) neutrinos. Solar fusion produces only electron neutrinos, but oscillations into other flavors during travel to Earth explain why early detectors registered a deficit.

The Sudbury Neutrino Observatory (SNO), 2 kilometers underground in Ontario, used heavy water to measure both total neutrino flux and electron neutrino flux. SNO's 2001 results confirmed the total flux matched predictions, but two-thirds of electron neutrinos had oscillated into other flavors en route to Earth.

Neutrino oscillations require nonzero mass. The original Standard Model assumed massless neutrinos, so oscillations constitute evidence for physics beyond it. Current measurements indicate neutrino masses are less than a few tenths of an electron volt — over a million times smaller than the electron mass.

This discovery resolved the solar neutrino problem and validated both solar fusion theory and quantum field theory. The neutrino flux matches predictions from nuclear burning models. Oscillations opened new physics avenues, including CP violation studies and implications for the universe's matter-antimatter asymmetry.

While neutrinos probe nuclear processes directly, helioseismology — the study of solar oscillations — maps conditions throughout the solar interior.

The Sun undergoes acoustic oscillations driven by outer-layer convection. These pressure waves propagate through the interior like seismic waves through Earth. Oscillation frequencies, amplitudes, and patterns depend on internal temperature, density, and composition profiles.

Solar oscillations appear as periodic Doppler shifts in photospheric absorption lines. The Global Oscillation Network Group (GONG) and Solar and Heliospheric Observatory (SOHO) monitor these oscillations continuously. Millions of distinct modes have been identified, each with characteristic radial and angular patterns.

Analyzing oscillation mode frequencies reveals the Sun's interior structure — a three-dimensional map of temperature, density, and rotation rate versus depth and latitude.

Helioseismology confirms stellar model predictions with high accuracy. Temperature profiles match theory within $0.1\%$ throughout most of the interior. The convective zone depth measures $0.287$ solar radii from the surface, matching theoretical predictions. It also validates density and temperature profiles used for neutrino predictions. The inferred central temperature of $(1.57 \pm 0.01) \times 10^7\,\text{K}$ confirms conditions for the observed proton-proton chain rate. This independent confirmation strengthens confidence in stellar evolution theory and solar nuclear processes.

\inlineimage{0.35}{10_SolarFusionQuantumTunneling/prisoner.png}{There’s a nonzero probability I’ll tunnel out.}
    \IfFileExists{#1/phenomenon_extra.tex}{\input{#1/phenomenon_extra}}{}
    
    % Include optional materials as part of the 5-page content block
    \IfFileExists{#1/joke.tex}{%
        \fullpagehumor[Folklore About Some Interesting Bugs]{%
\textbf{Geographic Email Limits} \\[4pt]
A university sysadmin faced a bizarre bug: emails wouldn't travel more than 500 miles. Boston at 420 miles worked fine; Memphis at 520 miles failed completely.\\
A timeout was set in nanoseconds instead of milliseconds. Light travels 200,000 km/s through fiber. The timeout expired at exactly 500 miles.\\[8pt]

\textbf{The Ice Cream Correlation} \\[4pt]
Drivers reported their keyless entry failed — but only after buying vanilla ice cream at one specific shop. Not chocolate, not strawberry. Just vanilla. Engineers were skeptical until they confirmed it.\\
The vanilla counter was at the front, chocolate in back. Vanilla buyers returned before the engine cooled, triggering a temperature-sensitive component failure.\\[8pt]

\textbf{Posture-Dependent Passwords} \\[4pt]
An IT team dismissed reports that a password worked standing but failed sitting — until they saw it happen. Same user, same password, different postures, different results.\\
Two keyboard keys had been swapped. Standing users looked down and typed what they saw. Seated users touch-typed from muscle memory, entering the wrong sequence.\\[8pt]

\textbf{Night Shift Crashes} \\[4pt]
Night-shift staff reported server crashes that never occurred during the day. No software changes, no power issues — just nightly failures.\\
The cleaning crew's floor polisher vibrated at 60 Hz, matching certain hard drives' resonant frequency. The vibration misaligned read/write heads just enough to crash the system.\\[8pt]

\textbf{The 2:45 PM Crash} \\[4pt]
One server crashed daily at exactly 2:45 PM. Logs showed thermal throttling, but only at that precise time.\\
Sunlight through a west-facing window hit the poorly ventilated rack at just the right angle. The temperature spike at 2:45 PM pushed the server past its limit.%
}
%
    }{}
    
    \IfFileExists{#1/exercises.tex}{%
        \fullpageexercises{%
\textbf{Energy Conservation Is an Amazing Tool: It Can Solve Seemingly Complex Problems Easily} \\[1em]
In classical mechanics, many systems that appear to require force analysis, Newton's laws, or torque computations can be solved using the principle of energy conservation. The following problems invite you to discover how straightforward solutions can emerge when total mechanical energy is conserved.

\vspace{1em}

\textbf{1. Rolling Sphere on an Incline} \\
A solid sphere of mass $m$ and radius $R$ is placed at the top of a frictional incline of angle $\theta$ and allowed to roll down without slipping. Using energy conservation (not Newton’s laws), determine the acceleration of the sphere’s center of mass. \\
\emph{Hint:} Total mechanical energy includes both translational and rotational kinetic energies.

\vspace{1em}

\textbf{2. Yo-Yo Drop} \\
A yo-yo of mass $m$ is held so that its string is taut and then released. The string unwinds without slipping as the yo-yo descends. The axle radius is $r$ and the moment of inertia about the center is $I$. Use energy conservation to determine the downward acceleration of the yo-yo’s center of mass. \\
\emph{Hint:} The yo-yo’s kinetic energy has both linear and rotational components; relate the angular speed to the linear speed via the axle radius.

\vspace{1em}

\textbf{3. Chain Falling Off a Table} \\
A uniform chain of linear mass density $\lambda$ lies coiled on a horizontal frictionless table. At time $t=0$, a small length starts to hang off the edge and the chain begins to slide off under gravity. Assuming no friction and no energy loss, use conservation of mechanical energy to find the acceleration of the chain as it falls. \\
\emph{Hint:} The center of mass of the hanging portion descends while its kinetic energy increases.

\vspace{1em}

\textbf{4. Man Walking on a Boat} \\
A man of mass $m$ walks a distance $d$ from one end of a boat of mass $M$ to the other. The boat floats on frictionless water. Determine how far and in what direction the boat moves relative to the water while the man walks. Use momentum conservation — no external horizontal forces act on the system. \\
\emph{Hint:} The center of mass of the system must remain fixed in the horizontal direction.


\noindent\hrulefill

\begin{center}
\rotatebox[origin=c]{180}{%
\begin{minipage}{0.95\textwidth}
\small
\noindent
\vspace{1em}
\(
\textbf{Answers: }\ 
\text{1: } a = \dfrac{5}{7}g\sin\theta \quad
\text{2: } a = \dfrac{mg}{m + \frac{I}{r^2}} \quad
\text{3: } a = \dfrac{g}{2} \quad
\text{4: } \text{Boat disp.} = -\dfrac{m}{M + m}d
\)
\end{minipage}
}
\end{center}


}
%
    }{}
    
    \IfFileExists{#1/cartoon.tex}{%
        \input{#1/cartoon}%
    }{}
    
    \IfFileExists{#1/imagefigure.tex}{%
        \input{#1/imagefigure}%
    }{}

    % --- PAGE 9: Technical (exactly 1 page) ---
    \newpage
    \begin{technical}
{\Large\textbf{Bioluminescence Mechanics and Quantification}}\\[0.2em]

\noindent\textbf{Molecular Reaction Pathway}\\
Firefly luciferase ($\sim 62\,\text{kDa}$, 550–560 amino acids) catalyzes a two-step reaction:
\begin{align*}
&\text{Luciferase} + \text{D-Luciferin} + \text{ATP} \notag\\
&\quad\rightarrow \text{Luciferase-luciferyl adenylate} + \text{PP}_i \notag\\
&\text{Luciferase-luciferyl adenylate} + \text{O}_2 \notag\\
&\quad\rightarrow \text{Oxyluciferin}^* + \text{AMP} + \text{CO}_2 \notag\\
&\text{Oxyluciferin}^* \notag\\
&\quad\rightarrow \text{Oxyluciferin} + \text{Light (560\,nm)}
\end{align*}

The emitted light has wavelength $\lambda$, corresponding to energy:

\begin{equation}
E = \frac{hc}{\lambda} = 
\frac{(6.626 \times 10^{-34}\,\text{J·s}) (3 \times 10^8\,\text{m/s})}
{560 \times 10^{-9}\,\text{m}} 
\end{equation}

\begin{equation}
\approx 3.55 \times 10^{-19}\,\text{J}
\end{equation}

Each reaction cycle consumes one ATP molecule. At most one photon can be emitted per catalytic turnover; the actual number of emitted photons per turnover equals the chemiluminescence quantum yield (typically on the order of 0.4–0.6 under physiological conditions). ATP drives formation of the luciferyl-adenylate intermediate; the photon energy primarily derives from subsequent oxidation of luciferin. The high quantum yield indicates minimal energy loss through non-radiative pathways.

\noindent\textbf{Cellular Organization}\\
The light-producing cells (photocytes) contain the necessary biochemical machinery:

\begin{itemize}[leftmargin=*,topsep=0pt,itemsep=0pt]
    \item Photocyte density: $10^5$–$10^6$ per lantern
    \item Luciferase molecules: $10^6$–$10^7$ per photocyte
    \item ATP concentration: 2–5 mM in photocytes
    \item Oxygen control: Regulated by nitric oxide signaling
\end{itemize}

Flash duration (200–300 ms) is controlled by oxygen availability via tracheal end cells that regulate gas diffusion to photocytes.

\noindent\textbf{Photon Count Measurements}\\
Modern spectrometer studies can be used to estimate counts of 
$3 \times 10^8$–$5 \times 10^8$ photons per flash. Given a flash duration of 200–300 ms, the temporal photon emission rate is:
\begin{align*}
\Phi_{\text{photons}} &\approx 
\frac{4 \times 10^8\,\text{photons}}{0.25\,\text{s}} \notag\\
&\approx 1.6 \times 10^9\,\text{photons/s}
\end{align*}

Converting to radiometric power:
\begin{align*}
P &= \Phi_{\text{photons}} \cdot E_{\text{photon}} \\
  &\approx (1.6 \times 10^9\,\text{s}^{-1}) \cdot (3.55 \times 10^{-19}\,\text{J}) \\
  &\approx 5.7 \times 10^{-10}\,\text{W}
\end{align*}

This power output, while small in absolute terms, appears significant to human observers due to perceptual factors.
\vspace{0.5em}

\noindent\textbf{Optical and Perceptual Optimization}\\
Firefly light appears significantly brighter than its raw power suggests due to: spectral peak near photopic sensitivity ($\sim$550$\,\text{nm}$), maximizing luminous efficacy; uric-acid crystal reflectors that direct internally scattered photons outward; surface texturing on the lantern that reduces interface reflection; and human dark adaptation, which increases apparent brightness in low light.

The luminous flux can be estimated as:
\begin{align*}
\text{Luminous }&\text{flux} = P \cdot 683\,\text{lm/W} \\
&\approx (5.7 \times 10^{-10}\,\text{W}) \cdot (683\,\text{lm/W}) \\
&\approx 3.7 \times 10^{-7}\,\text{lm}
\end{align*}

While this raw luminous flux is extremely low, perceptual effects amplify the apparent brightness by several orders of magnitude. 

\vspace{0.5em}
\noindent\textbf{References:}\\
{\footnotesize
Harvey, E. N. (1920). The Nature of Animal Light.\\
Wilson, T. \& Hastings, J.W. (1998). Bioluminescence. \textit{Annu. Rev. Cell Dev. Biol.}, \textbf{14}, 197-230.\\
Shimomura, O. (2012). \textit{Bioluminescence: Chemical Principles and Methods}. World Scientific.\\
}
\end{technical}

    % --- NO PAGE 10 EMPTY PAGE HERE ---
    % Empty verso pages are now added BEFORE each chapter (see top of \inputstory)
    % This prevents cumulative page drift and maintains 10-page alignment
    % Last chapter will not have trailing empty page


}

\definecolor{purplebackground}{RGB}{230,230,250} % Lavender/purple background
\newsavebox{\SideTextBox}
\newenvironment{SideNotePage}[2] % #1 = side text, #2 = image path
{%
  \clearpage
  \thispagestyle{empty}
  
  % Layout for 432x972pt PDFs within 5.5"×8.5" text area
  % Left: caption, vertical line separator, Right: centered image
  \vspace*{\fill}
  \noindent
  \begin{minipage}[c]{1.8in}
    \raggedright
    \footnotesize
    #1
  \end{minipage}%
  \hspace{0.1in}%
  \rule[-3.5in]{0.5pt}{7in}% vertical line separator
  \hspace{0.1in}%
  \begin{minipage}[c]{3.4in}
    \centering
    \includegraphics[width=3.2in,height=7.5in,keepaspectratio]{#2}
  \end{minipage}%
  \vspace*{\fill}
}
{%
  \clearpage
}





% Ensure technical environment doesn't break across pages
\AtEndEnvironment{technical}{\nopagebreak[4]}
