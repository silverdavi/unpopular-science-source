% Basic document setup
% 7“×10” format with appropriate margins
\usepackage[paperwidth=7in, paperheight=10in, margin=0.75in, inner=0.875in, outer=0.625in]{geometry}
\usepackage[latin]{babel}
\usepackage{graphicx}
\usepackage{booktabs}
\usepackage{epstopdf}
\usepackage{float}
\usepackage{tikz}
\usepackage{tikzsymbols}
\usepackage{pgfplots}
\usepackage{pgffor}
\usepackage{chemfig}
\usepackage{enumitem}
\usepackage{tabularx}
\usepackage{graphicx}
\usepackage{changepage}
\usepackage{verbatim}
\usepackage{comment}
\usepackage{setspace}
\usepackage{ragged2e}
\usepackage{array}
\usepackage{pgffor} % Required for \foreach
\usepackage{ifoddpage}
\usepackage{catchfile} 
\usepackage{xstring}
\usepackage[most]{tcolorbox}  % Ensure this is included in your preamble
\usepackage{array}
\usepackage{colortbl}
\usepackage{caption}
\usepackage{varwidth}
\usepackage{pagecolor}
\usepackage{lipsum} % optional for filler text
\usepackage{eso-pic} % For background picture in sidenotes



\pgfplotsset{compat=1.18}

% Typography and text formatting
\usepackage[protrusion=true,expansion=true]{microtype}
\usepackage{CJK}
\newcommand{\katakana}[1]{\begin{CJK}{UTF8}{min}#1\end{CJK}}
\usepackage{shadowtext}
\usepackage{bbding}
\usepackage{textcomp}
\usepackage{hyperref}
\usepackage{parskip}
\usepackage{multicol}
\usepackage{needspace}
\usepackage{etoolbox}
\usepackage{pifont}  % for nice symbols like the filled circle

% Math packages
\usepackage{amsmath, amssymb, physics}
\usepackage{unicode-math}

% Multilingual support
\usepackage{polyglossia}
\usepackage{amsfonts}
\usepackage{lmodern} 
\setdefaultlanguage{english}
\setotherlanguage{greek}
\setotherlanguage{hebrew}
\setotherlanguage{sanskrit}
\setotherlanguage{arabic}

% Assign fonts to polyglossia languages
\newfontfamily\greekfont[Script=Greek]{Linux Libertine O}
\newfontfamily\hebrewfont[Script=Hebrew]{Arial Hebrew}
\newfontfamily\sanskritfont[Script=Devanagari]{Devanagari Sangam MN}
\newfontfamily\arabicfont[Script=Arabic,Scale=1.2]{DecoType Naskh}

% Font configuration
\usepackage{fontspec}
\setmainfont{Libertinus Serif}                    % Main serif font
\newfontfamily\historyfont{Crimson Pro}           % Historical section font
          % Technical section font
\newfontfamily\commentaryfont{Crimson Pro}        % Commentary font


\newfontfamily\piefontfamily{Charis SIL}
\newfontfamily\ipafontfamily{Charis SIL}

% Create properly scoped commands that preserve current font size
\newcommand{\piefont}[1]{{\piefontfamily #1}}
\newcommand{\ipafont}[1]{{\ipafontfamily #1}}

\newfontfamily\summaryfont{Libertinus Serif Italic} % Summary font
\newfontfamily\technicalfont{Libertinus Serif}

% Color definitions
\usepackage{xcolor}
\definecolor{historycolor}{RGB}{70,30,0}         % Warm brown for historical sections
\definecolor{technicalcolor}{RGB}{0,20,60}       % Deep blue for technical sections
\definecolor{commentarycolor}{RGB}{0,0,0}        % Black for commentary
\definecolor{summarycolor}{RGB}{90,90,90}        % Gray for summaries
\definecolor{linkcolor}{RGB}{0,85,155}           % Link color for hyperlinks
\definecolor{SUMMARYCOLOR}{RGB}{90,90,90}        % Gray for summaries (uppercase)
\definecolor{lavender}{RGB}{230,230,250}
\definecolor{lightgray}{gray}{0.85}


% Section styling with titlesec
\usepackage{titlesec}
\usepackage{tocloft}  % For customizing TOC

\titleformat{\chapter}[hang]
    {\normalfont\huge\bfseries}{\thechapter}{1em}{} % Chapter title styling
\titlespacing*{\chapter}{0pt}{0pt}{10pt}            % Adjust spacing around chapters
\titleformat{\section}[hang]
    {\normalfont\Large\bfseries}{\thesection}{1em}{} % Section title styling
\titlespacing*{\section}{0pt}{10pt}{5pt}             % Adjust spacing around sections

% Frame and box packages
\usepackage[framemethod=TikZ,skipabove=6pt,skipbelow=6pt]{mdframed}
\usetikzlibrary{decorations.pathmorphing, decorations.shapes, decorations.footprints, shapes.geometric, positioning, patterns, fit,arrows.meta, decorations.pathmorphing, backgrounds, calc, decorations.fractals}
\mdfsetup{splitbottomskip=2pt, splittopskip=2pt}

\usetikzlibrary{lindenmayersystems}

% Spacing and layout settings
\setlength{\parindent}{0pt}      % No paragraph indentation
\setlength{\parskip}{0.5em}      % Paragraph spacing for better readability
\setlength{\columnsep}{15pt}     % Adjust column separation for multicol layout


% Force content to start on the next even-numbered page

\newcommand{\startchapter}{%
  \clearpage
  \checkoddpage
  \ifoddpage
    % already odd — do nothing
  \else
    \hbox{}
    \thispagestyle{empty}
    \clearpage
  \fi
}




% TOC customization for chapter summaries
\setlength{\cftbeforechapskip}{1.0em}  % Increase space between TOC entries
\setlength{\cftchapindent}{0em}        % Chapter indent in TOC
\renewcommand{\cftchapfont}{\bfseries} % Chapter font in TOC

% Header setup with fancyhdr - MUST be loaded after titlesec
\usepackage{fancyhdr}
\setlength{\headheight}{14pt}  % Slightly more than the required 13.59999pt

% Create chapter title extraction command to handle only the title part
\makeatletter
\def\extracttitle#1\\#2\@nil{#1}
\makeatother

% Define the main page style
\pagestyle{fancy}
\fancyhf{} % Clear all header and footer fields
\fancyhead[LE,RO]{\thepage}
\fancyhead[RE]{\textit{\leftmark}}
\fancyhead[LO]{\textit{\rightmark}}
\renewcommand{\headrulewidth}{0.4pt}
\renewcommand{\footrulewidth}{0pt}

% Fix the chapter page style (first page of each chapter)
\fancypagestyle{plain}{%
  \fancyhf{} % Clear all header and footer fields
  \fancyfoot[C]{\thepage} % Just page number at bottom
  \renewcommand{\headrulewidth}{0pt} % No header rule on first page
}

% Modify chaptermark to extract just the title
% Better approach to handle headers without the summary
\makeatletter
% Simple approach - standard chapter marking
\renewcommand{\chaptermark}[1]{%
  \markboth{\MakeUppercase{\chaptername\ \thechapter.\ #1}}{}}
\renewcommand{\sectionmark}[1]{\markright{\thesection.\ #1}}

% Helpers to input title and summary from file
\newcommand{\inputtitle}[1]{\IfFileExists{#1/title.tex}{Real Democracy Has Never Been Tried}{MissingTitle}}
\newcommand{\inputsummary}[1]{\IfFileExists{#1/summary.tex}{The exponential function extends far beyond calculus, appearing across mathematics as a bridge between local and global structure. From power series to Lie theory, from Riemannian geometry to sheaf cohomology, exponential maps carry additive or infinitesimal data into multiplicative, compositional, or curved settings. What began as a trick for quick computation has become a central map linking analysis, geometry, and algebra.}{MissingSummary}}

\newtcolorbox{shadedstory}[1][]{
  colback=gray!10,
  colframe=gray!50,
  sharp corners,
  boxrule=0.8pt,
  left=10pt,
  right=10pt,
  top=10pt,
  bottom=10pt,
  title=#1,
  fonttitle=\bfseries\large,
  enhanced,
}



\newcommand{\chapterseparator}{%
  \begin{center}
    % Include chapter-specific fractal tree PNG from fractal_trees/with_fruits/
    \includegraphics[height=3.5cm]{fractal_trees/with_fruits/\thechapter.png}
  \end{center}
}





% Custom chapter with summary command (TOC only, no visible title)
\newcommand{\chapterwithsummary}[3][]{%
  % #1 = optional label
  % #2 = chapter title
  % #3 = chapter summary

  % TOC entry only (multi-line)
  \addcontentsline{toc}{chapter}{%
    \protect\numberline{\thechapter}#2\\
    {\normalfont\small\textit{\textcolor{summarycolor}{#3}}}%
  }

  % Update header with title only (no page title)
  \chaptermark{#2}%

  % Apply label if provided
  \ifx\\#1\\\else\label{#1}\fi

  % Avoid rendering a visible chapter heading
  \phantomsection % for correct link targets
  \vspace*{-3em} % suppress visual whitespace
}


\newread\file
\newcommand{\readfirstline}[2]{%
  \openin\file=#2
  \read\file to \temp
  \closein\file
  \xdef#1{\temp}%
  % Ensure temp is not accidentally output
}
\newcommand{\chapterwithsummaryfromfile}[2][]{%
  \refstepcounter{chapter}
  \IfFileExists{#2/title.tex}{
    \readfirstline{\chaptertitle}{#2/title.tex}
  }{
    \def\chaptertitle{Missing Title}
  }
  \IfFileExists{#2/summary.tex}{
    \readfirstline{\chaptersummary}{#2/summary.tex}
  }{
    \def\chaptersummary{No summary available.}
  }
  
  % Store title and summary for later TOC entry and header (removed immediate TOC entry)
  \gdef\storedchaptertitle{\chaptertitle}
  \gdef\storedchaptersummary{\chaptersummary}
  
  % Do NOT set chaptermark here - delay until actual chapter content begins
  \ifx\\#1\\\else\label{#1}\fi
  \phantomsection
  \vspace*{-3em}
}

\makeatother

% Fix hyperref issues with multi-line TOC entries
\makeatletter
\patchcmd{\@chapter}
  {\addcontentsline{toc}{chapter}%
    {\protect\numberline{\thechapter}#1}}
  {\addcontentsline{toc}{chapter}%
    {\protect\numberline{\thechapter}#1}}
  {}{}
\makeatother

% Hyperlink settings - must be AFTER all other settings
\usepackage{hyperref}
\pdfstringdefDisableCommands{%
  \def\\{ }% Replace \\ with space in PDF strings
}
\hypersetup{
    colorlinks=true,
    linkcolor=linkcolor,
    citecolor=linkcolor,
    urlcolor=linkcolor
}

% Custom environments
% Historical section styling
\newenvironment{historical}{%
    \begin{mdframed}[%
        linecolor=historycolor, 
        linewidth=0.5pt, 
        backgroundcolor=white, 
        topline=false, 
        bottomline=false, 
        rightline=false, 
        leftmargin=1em, 
        rightmargin=1em]
    \historyfont\color{historycolor}
}{%
    \end{mdframed}
}

% Technical environment with improved multicols handling
\newenvironment{technical}{%
    \par\medskip\noindent
    \begin{tcolorbox}[
        enhanced,
        colframe=technicalcolor,
        colback=gray!5,
        boxrule=0.8pt,
        arc=0mm,
        left=5pt,
        right=5pt,
        top=5pt,
        bottom=10pt,
        breakable=false,
        parbox=false,
        after={\par\nobreak\noindent}  % Prevent page break after the box
    ]
    \begingroup % Begin a local group for scoping
    \renewcommand{\bfseries}{\fontseries{b}\selectfont} % Ensure bold works
    \renewcommand{\itshape}{\fontshape{it}\selectfont}  % Ensure italics works
    \renewcommand{\emph}[1]{\textit{##1}}               % Escape # properly
    \technicalfont\small % Apply technical font and small size
    \sloppy % Allow more flexible line breaking
    \hyphenpenalty=50 % Make hyphenation more likely
    \exhyphenpenalty=50 % Make hyphenation of already hyphenated words more likely
    
    % Adjust itemize spacing for technical environment
    \setlist[itemize]{nosep, leftmargin=1em, itemsep=0.1em, topsep=0.3em, partopsep=0pt}
    
    % Use multicols directly with adjusted parameters
    \begin{multicols}{2}
    \setlength{\columnseprule}{0.1pt} % Thin rule between columns
    \setlength{\columnsep}{12pt} % Slightly reduced column separation
}{%
    \end{multicols}
    \endgroup % End the local group, restoring global settings
    \end{tcolorbox}
}

% Commentary environment styling
\newenvironment{commentary}[1][Commentary]{%
    \bigskip % Add vertical space before commentary
    % Title with appropriate styling
    \begin{center}
        {\large\itshape\bfseries\textcolor{commentarycolor}{#1}}
    \end{center}
    \vspace{0.5em}
    % Content with right border
    \begin{mdframed}[%
        linecolor=commentarycolor, 
        linewidth=1pt, 
        backgroundcolor=white, 
        topline=false, 
        bottomline=false, 
        rightline=true, 
        leftmargin=1em, 
        rightmargin=1em]
    \commentaryfont\color{commentarycolor}
    \RaggedRight
}{%
    \end{mdframed}
}

\newcommand{\centeredfullpage}[2]{%
  \clearpage
  \thispagestyle{empty}

  \vspace*{\fill}

  \begin{center}
    \begin{minipage}{0.8\textwidth}
      \centering
      #1

      \vspace{1em}

      {\fontsize{12pt}{14pt}\selectfont \textbf{\textit{#2}}}
    \end{minipage}
  \end{center}

  \vspace*{\fill}

}



% Humor section for jokes and funny stories
\newenvironment{humorbox}[1][Lighter Side]{%
    \begin{figure}[b]  % Position at bottom of page by default
        \begin{tcolorbox}[
            enhanced,
            colframe=gray!50!black,
            colback=gray!5,
            boxrule=0.5pt,
            arc=2mm,
            left=10pt,
            right=10pt,
            top=6pt,
            bottom=6pt,
            title=#1,
            fonttitle=\bfseries\itshape,
            coltitle=black,
            breakable,
            parbox=false
        ]
        \itshape  % Make the text italic
}{%
        \end{tcolorbox}
    \end{figure}
}

\newcommand{\fullpagehumor}[2][Lighter Side]{%
  \clearpage
  \thispagestyle{empty}
  
  \vspace*{\fill}
  \noindent
  \begin{tcolorbox}[
      enhanced,
      width=\textwidth,
      colframe=gray!50!black,
      colback=gray!5,
      colbacktitle=gray!30,
      coltitle=black,
      boxrule=0.5pt,
      arc=2mm,
      left=15pt,
      right=15pt,
      top=10pt,
      bottom=10pt,
      title=#1,
      fonttitle=\bfseries\itshape
  ]
    \itshape #2
  \end{tcolorbox}
  \vspace*{\fill}
}



\newenvironment{exercisebox}[1][Exercises]{%
    \begin{tcolorbox}[
        enhanced,
        breakable,  % <-- Add this
        colframe=blue!60!black,
        colback=blue!5,
        boxrule=0.5pt,
        arc=2mm,
        left=15pt,
        right=15pt,
        top=5pt,
        bottom=10pt,
        title=#1,
        fonttitle=\bfseries,
        coltitle=white,
        colbacktitle=blue!60!black,
        attach boxed title to top left={yshift=-2mm, xshift=5mm},
        boxed title style={arc=1mm, boxrule=0.5pt}
    ]
    \begin{enumerate}[leftmargin=*]
}{%
    \end{enumerate}
    \end{tcolorbox}
}


% Full-page exercises section
\newcommand{\fullpageexercises}[2][Exercises]{%
  \clearpage                  % start it on a fresh page
  \thispagestyle{empty}       % no header/footer on page one of the box
  \begin{center}
    \begin{tcolorbox}[
      enhanced,
      breakable,               % < — allow the box to span pages
      width=\textwidth,
      colframe=blue!60!black,
      colback=blue!5,
      colbacktitle=blue!60!black,
      coltitle=white,
      boxrule=0.5pt,
      arc=2mm,
      left=5pt,
      right=5pt,
      top=10pt,
      bottom=10pt,
      title=#1,
      fonttitle=\bfseries,
      attach boxed title to top left={yshift=-2mm, xshift=5mm},
      boxed title style={arc=1mm, boxrule=0.5pt}
    ]
      #2
    \end{tcolorbox}
  \end{center}
  % **no** \clearpage or \cleardoubleoddpage here
}



\newcommand{\imagefigure}[2]{%
  \thispagestyle{empty}
  \begin{center}
    \vspace*{2em} % Small spacing instead of \null\vfill
    #1

    \vspace{1em}

    % Caption (only if nonempty)
    % Adjusted margins for 7“×10” format (was 2cm each side)
    \begin{adjustwidth}{1.5cm}{1.5cm}
      {\fontsize{11pt}{14pt}\selectfont #2}
    \end{adjustwidth}

  \end{center}
  \vspace{2em}
}


% Image with caption vertically centered in remaining space
% Usage: \inlineimage{width_factor}{path/to/image.png}{Caption text}
% Example: \inlineimage{0.8}{image.png}{My caption} uses 0.8\linewidth
\newcommand{\inlineimage}[3]{%
  \par
  \vspace{0pt plus 1fill}% Flexible space above for vertical centering
  \begin{center}
    \includegraphics[width=#1\linewidth,keepaspectratio]{#2}\\[0.5em]
    {\small\textit{#3}}
  \end{center}
  \vspace{0pt plus 1fill}% Flexible space below for vertical centering
}


% Topic map macro: render as small caps with ○ separator
\newcommand{\topicmap}[1]{%
  \noindent\textsc{%
    \def\tempa{}%
    \foreach \x in {#1} {%
      \ifx\tempa\empty
        \x
      \else
        \enspace$\circ$\enspace\x
      \fi
      \xdef\tempa{nonempty}%
    }%
  }%
}




\newcommand{\storyintro}[1]{%
  \clearpage
  \thispagestyle{empty}
  \begin{center}
    \vspace*{\fill}

    % Title
    {\Huge \bfseries Real Democracy Has Never Been Tried}

    \vspace{1em}

    % Summary
    % Summary (cleaner, no forced italics)
    \vspace{1em}
    \begin{minipage}{0.75\textwidth}
        \centering
        {\Large \color{summarycolor} The exponential function extends far beyond calculus, appearing across mathematics as a bridge between local and global structure. From power series to Lie theory, from Riemannian geometry to sheaf cohomology, exponential maps carry additive or infinitesimal data into multiplicative, compositional, or curved settings. What began as a trick for quick computation has become a central map linking analysis, geometry, and algebra.}
    \end{minipage}


    \vspace{1.5em}

    % Separator
    {\Large \textcolor{gray}{\ding{108}}}

    \vspace{1em}

    % Topicmap
    {\normalsize \textsc{\topicmap{
Black Hole Event Horizon,
Schwarzschild Radius,
Time Becomes Spacelike,
Singularity as Future,
LIGO Gravitational Waves,
EHT M87 Image,
Coordinate Dependencies,
Penrose-Hawking Theorems,
Kerr's Quora Objections,
White Holes \& Wormholes,
$r$ Becomes Timelike
}}}

    \vspace*{\fill}
  \end{center}
  \clearpage
}





\newcommand{\inputstory}[1]{%
    % === CHAPTER STRUCTURE: EXACTLY 10 PAGES ===
    % 1. Big Title (recto) - 1 page
    % 2. Sidenote (verso) - 1 page  
    % 3. Title+Summary+Topicmap+Quote (recto) - 1 page
    % 4-8. Historical+Main content - 5 pages
    % 9. Technical - 1 page
    % 10. Empty verso page - 1 page (for chapter separation)
    %
    % VERSO EMPTY PAGE BEFORE EACH CHAPTER (including first)
    % This ensures proper recto/verso alignment and 10-page structure
    % Every chapter gets a verso separator page for consistent alignment
    \clearpage
    \thispagestyle{empty}
    \mbox{}
    \clearpage
    
    % Start chapter on new page - chapter numbering handled by \chapterwithsummaryfromfile
    \clearpage

    % --- PAGE 1: Dedicated Title Page (big, centered) ---
    \phantomsection
    \readfirstline{\chaptertitle}{#1/title.tex}
    \readfirstline{\chaptersummary}{#1/summary.tex}

    \thispagestyle{empty}
    \begin{center}
        \vspace*{\fill}
        {\fontsize{48pt}{62pt}\selectfont\bfseries\raggedright
        \parbox{0.8\textwidth}{\centeringReal Democracy Has Never Been Tried}}
        \vspace*{\fill}
    \end{center}

    \clearpage

    % --- PAGE 2: Sidenote ---
    \IfFileExists{#1/sidenote.tex}{%
        \begin{SideNotePage}{
  \textbf{Top (Voting Methods):} \par Ranked-choice voting is a system in which voters express their preferences by submitting complete rankings of all candidates, and the system aggregates them into a ranked list or a single winner. Different aggregation methods (Borda count, IRV, Plurality, Condorcet) can produce distinct winners from identical voter rankings, demonstrating the inherent ambiguity in collective decision-making. The same preference profile can yield different outcomes depending on which features of the rankings are emphasized by the chosen method. This indeterminacy reveals that there is no canonical way to translate individual preferences into collective choices.


  \textbf{Bottom (Arrow's Theorem):} \par Arrow's impossibility theorem proves that no ranked-choice voting method can satisfy all four fairness criteria simultaneously when there are at least three alternatives and two voters. Every democratic aggregation method must compromise at least one criterion, making trade-offs unavoidable in social choice. 
}{09_ArrowTheoremTopology/09_ Real Democracy Has Never Been Tried.pdf}
\end{SideNotePage}%
    }{%
        % If no sidenote, create empty page
        \thispagestyle{empty}
        \mbox{}
    }{}

    \clearpage

    % Add TOC entry HERE (after sidenote page, to show correct page in PDF TOC)
    \addcontentsline{toc}{chapter}{%
      \protect\numberline{\thechapter}\storedchaptertitle\\
      {\normalfont\small\textit{\textcolor{summarycolor}{\storedchaptersummary}}}%
    }

    \clearpage

    % --- PAGE 3: Title + Summary + Topicmap + Quote (original intro style) ---
    \thispagestyle{empty}
    \begin{center}
        \vspace*{\fill}

        {\Huge \bfseries Real Democracy Has Never Been Tried}

        \vspace{2em}
        \begin{minipage}{0.8\textwidth}
            {\fontsize{13pt}{18pt}\selectfont\color{black}
            \justifying
            The exponential function extends far beyond calculus, appearing across mathematics as a bridge between local and global structure. From power series to Lie theory, from Riemannian geometry to sheaf cohomology, exponential maps carry additive or infinitesimal data into multiplicative, compositional, or curved settings. What began as a trick for quick computation has become a central map linking analysis, geometry, and algebra.}
        \end{minipage}

        \vspace{2em}

        \chapterseparator

        \vspace{2em}

        \IfFileExists{#1/topicmap.tex}{%
            \begin{minipage}{0.7\textwidth}
                \centering
                 \topicmap{
Black Hole Event Horizon,
Schwarzschild Radius,
Time Becomes Spacelike,
Singularity as Future,
LIGO Gravitational Waves,
EHT M87 Image,
Coordinate Dependencies,
Penrose-Hawking Theorems,
Kerr's Quora Objections,
White Holes \& Wormholes,
$r$ Becomes Timelike
}
            \end{minipage}
        }{}

        \vfill

        \IfFileExists{#1/quote.tex}{%
            \vspace{2em}
            \begin{minipage}{0.8\textwidth}
                \centering \itshape \begin{flushright}
\emph{„Ich bin überzeugt, dass der Alte nicht würfelt.“}\\
\emph{("I am convinced that God does not play dice.")}\\
— Albert Einstein, 1926
\end{flushright}

\vspace{2em}

\begin{flushright}
\emph{"Hör auf, Gott vorzuschreiben, was er zu tun hat!"}\\
\emph{("Stop telling God what to do!")}\\
— Niels Bohr, circa 1927
\end{flushright}

            \end{minipage}
        }{}

        \vspace*{\fill}
    \end{center}

    \clearpage

    % --- PAGES 4-8: Historical + Main + Optional materials (exactly 5 pages total) ---
    % Set chapter header HERE when content actually begins
    \chaptermark{\storedchaptertitle}
    {\LARGE \bfseries Real Democracy Has Never Been Tried}
    \begin{historical}
Pirkei Avot opens with the chain of tradition: “Moses received the Torah from Sinai and transmitted it to Joshua, and Joshua to the Elders, and the Elders to the Prophets, and the Prophets transmitted it to the Men of the Great Assembly.”

This transmission of authority defined who could determine Jewish law, including calendar matters. The chain continued through specific named authorities: Shimon the Righteous (one of the last of the Great Assembly), Antigonus of Socho, then paired leaders through the generations — the Zugot (pairs), where one served as Nasi (president) and one as Av Beit Din (head of the court).

The pairs included Yose ben Yoezer and Yose ben Yochanan, Joshua ben Perachya and Nittai of Arbel, Judah ben Tabbai and Shimon ben Shetach, Shemaya and Avtalyon, and finally Hillel and Shammai. From Hillel descended a dynasty of leaders who held the title of Nasi through the destruction of the Second Temple in 70 CE and beyond.

During the Temple period, this leadership controlled calendar determination. The Sanhedrin, with the Nasi presiding, declared new months based on witness testimony and intercalated years to maintain seasonal alignment. Their authority to declare time derived from the biblical verse “These are the appointed seasons of the Lord, which you shall proclaim” — the Hebrew emphasizes “which YOU shall proclaim,” granting human authorities the power to establish sacred time.

After the Temple's destruction in 70 CE, the Sanhedrin reconvened in Yavneh under Rabban Yochanan ben Zakkai, then moved through various Galilean cities: Usha, Shefar'am, Beit She'arim, Sepphoris, and finally Tiberias. Despite lacking a Temple, they maintained calendar authority through the traditional chain of ordination (semicha) that connected each generation back to Moses.

The Roman Empire increasingly restricted Jewish self-governance. Emperor Hadrian outlawed ordination after the Bar Kokhba revolt (132-135 CE). Constantine I (306-337 CE) further limited Jewish courts' jurisdiction.

Hillel II served as Nasi from approximately 320 to 385 CE. Facing intensifying persecution and the imminent collapse of centralized Jewish authority, he made an unprecedented decision around 358 CE: publish the mathematical secrets of calendar calculation.
\end{historical}
    Digital cellular networks are designed to carry simultaneous conversations — across a limited radio spectrum. Each call consists of two independent data streams, one uplink and one downlink, connecting handset and base station. These streams must be separated both by directionality and from the traffic of other users occupying the same band.

So we have duplexing and multiplexing. Duplexing separates the uplink (phone to tower) from the downlink (tower to phone). In frequency division duplexing (FDD), each direction is assigned its own frequency band, allowing simultaneous transmission and reception. In time division duplexing (TDD), both directions share a common band but alternate in fixed, synchronized time slots.

Multiplexing separates users sharing the same physical channel. In frequency division multiple access (FDMA), the spectrum is divided into separate frequency bands, each assigned to a different user. This isolates signals but requires fixed bandwidth allocation and limits how flexibly users can be added or removed. In time division multiple access (TDMA), users transmit in alternating time slots within a repeating frame structure. Each user has exclusive access to the channel during its assigned slot. This improves spectral efficiency, but requires strict global timing to keep transmissions aligned. In code division multiple access (CDMA), all users transmit simultaneously over the same frequency band, but each encodes its data using a unique pseudorandom spreading code. The receiver uses correlation to extract the intended signal. This allows full-time transmission with statistical multiplexing, but demands signal separation. All three approaches require that each user's transmission be confined to a fixed envelope, a burst, with predictable alignment and duration.

These requirements propagate upward through the entire transmission stack. Each burst must arrive in its designated slot, with precise size and timing. Modulation, equalization, and error correction depend on this regularity. As a result, every upstream layer, from speech encoding to encryption, must preserve the burst format.

To transmit speech, the analog signal is sampled at regular intervals and each sample is encoded as a digital number. This raw bitstream is then compressed using a speech codec, a specialized algorithm that reduces bandwidth by representing only perceptually important features. As a toy example, consider a 20 ms segment of audio. Uncompressed, this might require over 2,000 bits. A codec might instead describe it using only pitch, volume, and phoneme class, reducing the bitrate by an order of magnitude.

GSM (the Global System for Mobile Communications) was developed as a pan-European standard for digital cellular networks in the early 1990s. It replaced earlier analog systems with a structured, time-synchronized digital stack designed for interoperability, moderate confidentiality, and efficient spectrum use. The GSM radio interface is based on TDMA: each 200 kHz carrier is divided into repeating time frames of eight slots, with each user assigned one slot per frame. Each slot (or burst) carries 114 bits of payload, framed by synchronization and guard bits.

Voice is transmitted as a sequence of such bursts. Every 20 milliseconds of speech is compressed into a 260-bit frame, which after coding and interleaving is split across multiple 114-bit radio bursts for transmission. These bits are divided into classes by perceptual importance. The most critical will later be protected with more redundancy. Each frame is processed independently and must be transmitted in order, aligned to the caller’s assigned slot. From this point forward, it is treated as a fixed-length atomic unit: encoded, encrypted, and modulated as a whole. Control channels like SACCH (Slow Associated Control Channel) follow a similar pattern, expanding 184-bit messages to 456 bits after error correction coding.

Before transmission, the frame is convolutionally encoded (which is a type of error correction coding). This adds redundancy by producing each output bit as a function of the current and previous inputs. The goal is to enable error correction at the receiver without retransmission. After encoding, the output is interleaved. It is reordered across time so that localized bit corruption does not overwhelm any one frame. These operations are deterministic and standardized. Their result is a longer, structured bitstream with predictable relationships between positions.

At this point, the data must be encrypted, but without affecting its size or timing. Each burst has a fixed payload size, and must be transmitted precisely at its assigned interval. This rules out modes that expand input or require buffering; encryption must operate in place with no change to length or alignment. GSM therefore uses a stream cipher: a keystream is generated and XORed with the data bit-for-bit, producing ciphertext of equal length and immediate readiness for modulation.

GSM fixes the processing order as: compression $\rightarrow$ error correction $\rightarrow$ interleaving $\rightarrow$ encryption. This sequencing is a deliberate engineering decision. By placing encryption at the end of the stack, the system isolates cryptographic logic from earlier processing stages. Each module performs a self-contained transformation. This design simplifies implementation — but, as we will see, introduces a vulnerability.

By the time the bitstream reaches the cipher, it is no longer raw data. It has been processed into a rigid format defined by the protocol. Within the 114 ciphered bits per burst this includes:

\begin{itemize}
  \item \textbf{Padding and known link-layer fields:} deterministic bits that fill or structure payloads on certain channels.
  \item \textbf{Error-correction codes:} parity bits computed from public polynomials.
  \item \textbf{Interleaving:} a known permutation applied identically to each block.
\end{itemize}

Each 114-bit burst contains payload data bracketed by tail, training, and guard intervals of fixed length; those bracket fields are not ciphered. Within the ciphered payload, the bitstream is heavily preprocessed prior to encryption. Bit patterns arising from coding, interleaving, and protocol padding are defined explicitly by the standard and repeat across sessions. The plaintext entering the encryption algorithm is therefore predictable at specific locations. It is drawn from a constrained distribution with high predictability and low entropy in fixed subregions. A passive observer capturing encrypted GSM traffic receives ciphertext derived from partially labeled inputs whose positions and formats are specified in advance by the protocol.

GSM's stream cipher preserves structure in a way that a block cipher with diffusion would not. Because encryption is bitwise XOR, linear relations introduced by coding survive intact in the ciphertext. Consider a concrete example: suppose the channel coding introduces a parity check — a known XOR relation among data bits. After encryption, the corresponding ciphertext bits satisfy the same parity relation among their respective keystream values. An attacker can deduce this constraint without knowing the underlying data.

By collecting multiple ciphertext samples, each reflecting similar patterns but different keystream realizations, the attacker builds a system of equations that gradually reduces the candidate key space. GSM compounds this vulnerability: voice frames are transmitted redundantly across multiple bursts, providing numerous ciphertext instances derived from aligned inputs. Repeated encipherment of predictable structure with the same key makes the keystream a target for mathematical reconstruction.

This vulnerability was exploited explicitly in the work of Eli Biham, Elad Barkan, and Nathan Keller. In 2003, they demonstrated a ciphertext-only attack against A5/2 capable of recovering the full 64-bit session key in under one second, using multiple frames of intercepted communication from control channels like SACCH. The attack made no assumptions about plaintext content beyond its adherence to GSM’s format. The weakness resulted from applying error correction and interleaving before encryption, allowing algebraic methods to exploit the resulting regularity. The attack combined bruteforce enumeration of the cipher's R4 register ($2^{16}$ possibilities) with solving overdetermined systems of linear equations derived from keystream parity constraints. This required hours of preprocessing and gigabytes of storage but was tractable on standard computing hardware.

In the same year, the authors presented an active attack that used this weakness in A5/2 to compromise A5/1 (a stronger cipher that GSM uses by default). GSM allows the base station to select the cipher for communication. A rogue station can impersonate a valid tower and request a downgrade to A5/2 from a handset that supports it. Once the device complies, the attacker captures the A5/2-encrypted exchange, recovers the session key, and then uses that key to decrypt subsequent bursts sent using A5/1. This is possible because GSM reuses the session key across ciphers during a session. The presence of A5/2 in the cipher suite thus undermines A5/1, regardless of whether the latter is ever explicitly requested by the attacker. Any device that implements A5/2 inherits its vulnerabilities and propagates them to the stronger cipher via shared key state.

Barkan and Biham continued to refine their attacks. In 2005, they improved known-plaintext techniques against A5/1, specifically targeting its irregular initialization procedure. This reduced the computational burden of recovering internal state, particularly in scenarios with limited plaintext exposure. However, their most significant advance came in 2006, when they extended ciphertext-only techniques to A5/1 itself. The approach required far more ciphertext and offline preprocessing than the attack on A5/2, but the principle was similar. By leveraging the publicly known convolutional codes used before encryption, the attackers extracted algebraic relations between ciphertext bits and the keystream. These relations were then used to filter candidate internal states of the cipher’s LFSRs (Linear Feedback Shift Registers), narrowing the search space to feasible dimensions. The complexity of the attack remained high, but it fell within the capabilities of a moderately resourced organization with access to terabyte-scale storage and standard computational infrastructure.

The attack on A5/1 demonstrated that GSM’s vulnerability was a results of the interplay between encryption placement, channel configuration, and cipher reuse. GSM’s decision to support multiple ciphers without enforcing mutual isolation of key state allowed one weak algorithm to compromise the integrity of the entire suite. Because GSM does not authenticate base stations, handsets cannot verify that cipher selection is legitimate. Any device supporting A5/2 remains exposed to downgrade. Once the session key is recovered through a break of A5/2 — whether using algebraic decoding or parity-based keystream reconstruction — that same key grants access to A5/1-protected content. GSM’s cipher suite is therefore not modular. Its effective security is determined not by the strongest cipher in use, but by the weakest that is supported. A5/2’s inclusion rendered A5/1 susceptible by transitive failure.
\newpage
\begin{commentary}[Protocol Assumptions and Personal Entry Point]

The weakest points in deployed cryptographic systems are rarely in the mathematics. They are in the layers that surround it: in protocol assumptions, state handling, framing conventions, or timing logic. This is why cryptographic standards are slow to change — not because better ciphers are unavailable, but because known, tested flaws are often safer than untested replacements. The defensive posture of a system is not just algorithmic strength, but mostly accumulated knowledge of how it fails.

I first encountered this issue in a lecture by Eli Biham around 2003. He outlined the GSM vulnerability using nothing but XOR equations, known plaintext segments, and short recurrence relations. This attack did not require knowing the full formalism of block cipher construction or number theory. It showed that security could collapse under regularity exposed by the protocol — and that the analysis of “where” in a system encryption occurs mattered as much as “how.”

\end{commentary}

\inlineimage{0.8}{12_GSMEncryptionOrder/Cellular_network_standards_and_generation_timeline.svg.png}{Cellular network standards and generation timeline, CC BY-SA 4.0, by Wikimedia Commons}
    \IfFileExists{#1/phenomenon_extra.tex}{\input{#1/phenomenon_extra}}{}
    
    % Include optional materials as part of the 5-page content block
    \IfFileExists{#1/joke.tex}{%
        \fullpagehumor[{Radio Yerevan Jokes}]{

    \textbf{Q: Radio Yerevan was asked: "Capitalism is the exploitation of man by man. What about Socialism?"} \\  
    \textbf{A:} Radio Yerevan answered: "Under Socialism it is exactly the other way around."  

    \medskip  
    \textbf{Q: Is there freedom of speech in the Soviet Union the same as in the USA?} \\  
    \textbf{A:} In principle, yes. In the USA, you can stand in front of the Washington Monument and yell, "Down with Reagan!" and you will not be punished. In the Soviet Union, you can stand in Red Square and yell, "Down with Reagan!" and you will not be punished.  

    \medskip  
    \textbf{Q: What is chaos?} \\  
    \textbf{A:} We do not comment on national economics.  

    \medskip  
    \textbf{Q: Is it true that half of the members of the Central Committee are fools?} \\  
    \textbf{A:} What a crazy question. Half of the members of the Central Committee are not fools!  

    \medskip  
    \textbf{Q: Is it true that Ivan Ivanovich Ivanov from Moscow won a car in a lottery?} \\  
    \textbf{A:} In principle, yes. But: it wasn't Ivan Ivanovich Ivanov but Aleksander Aleksandrovich Aleksandrov; he is not from Moscow but from Odessa; it was not a car but a bicycle; and he didn’t win it — it was stolen from him.  

    \medskip  
    \textbf{Q: Why is our government not in a hurry to land our men on the moon?} \\  
    \textbf{A:} What if they refuse to return?  

    \medskip  
    \textbf{Q: We are told that communism is already visible on the horizon. What then is a horizon?} \\  
    \textbf{A:} An imaginary line that moves farther away each time you approach it.  

    \medskip  
    \textbf{Q: Is it true that conditions in our labor camps are excellent?} \\  
    \textbf{A:} In principle, yes. Five years ago, one of our listeners was skeptical, so he was sent to investigate. He must have liked it so much that he hasn’t returned yet.  

    \medskip  
    \textbf{Q: What is the most beautiful city in the Soviet Union?} \\  
    \textbf{A:} Yerevan, obviously.\\
    \textbf{Q: How many nuclear bombs will it take to destroy Yerevan?} \\  
    \textbf{A:} Baku is also a beautiful city.  

    \medskip  
    \textbf{Q: Is it true that the United States is on the edge of a precipice?} \\  
    \textbf{A:} True, but we are a step farther than them.  

    \medskip  
    \medskip  
    \textbf{Q: Where do children grow up healthy, happy, and confident about their bright and wonderful future?} \\  
    \textbf{A:} In the Soviet Union!  
    \\(At this moment, cheering is interrupted by the sound of a child crying.)\\
    \textbf{Q: Verochka, why are you crying?} \\  
    \textbf{A:} I want to live in the Soviet Union...  

}
%
    }{}
    
    \IfFileExists{#1/exercises.tex}{%
        \fullpageexercises{%

\textbf{Fusible Numbers: Exercises in Constructive Time} \\

A classic riddle: given two candles, each burn for 1h, how can you measure 45m?
\vspace{1em}

Fusible numbers form a well-ordered subset of the rationals constructed iteratively from zero. A number \( z \) is fusible if there exist previously constructed fusible numbers \( x \) and \( y \), with \( |x - y| < 1 \), such that $z = (x + y + 1)/2$.
The construction corresponds to lighting a unit-time fuse at both ends with delay. Results about the growth of certain associated functions are linked to very fast-growing hierarchies and have connections to independence from Peano arithmetic.

\vspace{1em}
\textbf{1. The Fuse Construction} \\
For a unit fuse lit at time $x$ on one end and time $y$ on the other (with $|x-y| < 1$), prove it extinguishes at $z = (x + y + 1)/2$. Consider the burn dynamics when both flames are active.

\vspace{1em}
\textbf{2. Enumeration Below 2} \\
Determine all fusible numbers $<2$ by systematic application of the construction rule.

\vspace{1em}
\textbf{3. Dyadic Structure} \\
Prove that all fusible numbers have the form $a/2^k$ for integers $a \geq 0$ and $k \geq 0$.

\vspace{1em}
\textbf{4. The Margin Function} \\
Let $a_n$ be the smallest fusible number exceeding $n$. Prove that
\[
a_n = n + \frac{1}{2^{k(n)}}
\]
for some $k(n) \in \mathbb{N}$. Compute $k(0)$, $k(1)$, $k(2)$. The function $k(n)$ grows extremely rapidly; certain asymptotic properties are connected to statements independent of Peano arithmetic.

\vspace{1em}
\textbf{5. Well-Ordering} \\
Prove that the fusible numbers form a well-ordered subset of $\mathbb{Q}^+$. What does this imply about infinite decreasing sequences?

\vspace{2em}
\hrule
\vspace{1em}
\textbf{Context} \\
Fusible numbers (Erickson, Xu) demonstrate how elementary constructions yield extremely fast growth. The margin values are:
\[
a_0 = \frac{1}{2}, \quad a_1 = 1 + \frac{1}{8}, \quad a_2 = 2 + \frac{1}{1024}
\]
The growth behavior connects to ordinal $\varepsilon_0$. (See the chapter about big numbers)
}
%
    }{}
    
    \IfFileExists{#1/cartoon.tex}{%
        \input{#1/cartoon}%
    }{}
    
    \IfFileExists{#1/imagefigure.tex}{%
        \input{#1/imagefigure}%
    }{}

    % --- PAGE 9: Technical (exactly 1 page) ---
    \newpage
    \begin{technical}
{\Large\textbf{False Vacuum Decay: Mathematical Formulation}}\\[0.3em]

\textbf{Higgs Potential and Vacuum Stability}\\[0.5em]
The Higgs potential in the Standard Model takes the form
$$
V(\phi) = \mu^2 \phi^2 + \lambda \phi^4,
$$
where $\phi$ is the Higgs field, $\mu^2 < 0$ for spontaneous symmetry breaking, and $\lambda > 0$ for stability. The vacuum expectation value is $\langle \phi \rangle = v = \sqrt{-\mu^2/\lambda} \approx 246$ GeV.

However, renormalization group running modifies the effective potential at high energies. The quartic coupling evolves as
\begin{align*}
(16\pi^2)\,\beta_\lambda &= 12\lambda^2 + (12 y_t^2 - 9 g^2 - 3 g'^2)\lambda \\
&\quad - 12 y_t^4 + \frac{9}{8} g^4 + \frac{3}{4} g^2 g'^2 + \frac{3}{8} g'^4,
\end{align*}
where $y_t$ is the top quark Yukawa coupling, $g$ and $g'$ are the SU(2)$ _L$ and U(1)$ _Y$ gauge couplings, and $Q$ is the energy scale. For Higgs mass $m_H \approx 125$ GeV and top mass $m_t \approx 173$ GeV, $\lambda$ runs negative at scales $Q \sim 10^{10}$--$10^{11}$ GeV, creating a second minimum at large field values.

\textbf{Coleman-De Luccia Instanton}\\[0.5em]
Vacuum decay proceeds via bubble nucleation described by the Euclidean action
$$
S_E = \int d^4x \left[\frac{1}{2}(\partial_\mu \phi)^2 + V(\phi)\right].
$$
The critical bubble solution has $O(4)$ symmetry in Euclidean space, satisfying
$$
\frac{d^2\phi}{d\rho^2} + \frac{3}{\rho}\frac{d\phi}{d\rho} = \frac{dV}{d\phi},
$$
where $\rho = \sqrt{x_1^2 + x_2^2 + x_3^2 + x_4^2}$ is the four-dimensional radius.

The nucleation rate per unit volume is
$$
\Gamma = A e^{-S_E/\hbar},
$$
where $A$ is a prefactor and $S_E$ is the Euclidean action of the bounce solution. For the Standard Model, current estimates give $S_E/\hbar \sim 400-500$, making spontaneous decay negligible over cosmic timescales.

\textbf{High-Energy Triggers}\\[0.5em]
Local few-particle collisions (in colliders or from ultra-high-energy cosmic rays) are not expected to nucleate the required $O(4)$-symmetric critical bubble. Observed cosmic rays reach energies up to $\sim 3\times 10^{20}$ eV without any indication of catalyzed vacuum decay, consistent with the nonperturbative, extended nature of the tunneling process.

\textbf{Bubble Dynamics}\\[0.5em]
Once nucleated, the bubble wall accelerates due to the pressure difference between vacua. The wall Lorentz factor obeys
$$
\gamma^2 = \frac{1}{1-v^2},
$$
where $v$ is the wall velocity. In the thin-wall limit, the pressure difference $\epsilon$ across the wall drives $v$ rapidly toward the speed of light ($v \to 1$) as the bubble expands; the detailed dynamics depend on the surface tension $\sigma$, the energy difference $\epsilon$, and the background spacetime.

\textbf{Renormalization Group Uncertainty}\\[0.5em]
The stability analysis depends critically on precise measurements of Standard Model parameters. The most sensitive quantities are:
\begin{align*}
m_t &= 173.1 \pm 0.9 \text{ GeV} \\
m_H &= 125.25 \pm 0.17 \text{ GeV} \\
\alpha_s(M_Z) &= 0.1179 \pm 0.0010
\end{align*}
An increase of order $\sim$1 GeV in the top mass would shift the stability assessment toward instability, while a $\sim$3 GeV increase in the Higgs mass would favor absolute stability.

\vspace{0.5em}
\textbf{References:}\\
{\footnotesize
Coleman \& De Luccia, \textit{Phys. Rev. D} \textbf{21}, 3305 (1980).\\
Degrassi et al., \textit{JHEP} \textbf{08}, 098 (2012).
}
\end{technical}


    % --- NO PAGE 10 EMPTY PAGE HERE ---
    % Empty verso pages are now added BEFORE each chapter (see top of \inputstory)
    % This prevents cumulative page drift and maintains 10-page alignment
    % Last chapter will not have trailing empty page


}

\definecolor{purplebackground}{RGB}{230,230,250} % Lavender/purple background
\newsavebox{\SideTextBox}
\newenvironment{SideNotePage}[2] % #1 = side text, #2 = image path
{%
  \clearpage
  \thispagestyle{empty}
  
  % Layout for 432x972pt PDFs within 5.5“×8.5” text area
  % Left: caption, vertical line separator, Right: centered image
  \vspace*{\fill}
  \noindent
  \begin{minipage}[c]{1.8in}
    \raggedright
    \footnotesize
    #1
  \end{minipage}%
  \hspace{0.1in}%
  \rule[-3.5in]{0.5pt}{7in}% vertical line separator
  \hspace{0.1in}%
  \begin{minipage}[c]{3.4in}
    \centering
    \includegraphics[width=3.2in,height=7.5in,keepaspectratio]{#2}
  \end{minipage}%
  \vspace*{\fill}
}
{%
  \clearpage
}





% Ensure technical environment doesn't break across pages
\AtEndEnvironment{technical}{\nopagebreak[4]}
