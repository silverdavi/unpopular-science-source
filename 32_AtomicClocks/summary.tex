Timekeeping has progressively moved toward smaller physical phenomena: from Earth's rotation to pendulums, from crystal oscillations to atomic transitions, and now toward nuclear resonances. The SI second, defined by 9,192,631,770 periods of cesium-133's hyperfine transition, relies on quantum interactions between nuclear and electronic magnetic moments. This shift to microscopic reference standards improves precision exponentially — hydrogen masers achieve stability of 1 part in $10^{13}$, while optical lattice clocks using strontium reach 1 part in $10^{18}$ by probing transitions at ~$10^{15}$ Hz. The progression continues toward nuclear clocks using thorium-229, which promises precision of 1 part in $10^{19}$ by exploiting transitions in atomic nuclei rather than electron shells.
