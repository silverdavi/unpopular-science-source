\begin{historical}
In the mid-20th century, formal models of collective decision-making began to draw the attention of economists, political theorists, and mathematicians. Rather than treating voting as a procedural artifact, researchers sought to characterize what could or could not be achieved when individual preferences are aggregated into a group decision.

Kenneth Arrow’s work in the early 1950s became a cornerstone of this approach. During the following decades, related work by Allan Gibbard and Mark Satterthwaite showed that even the absence of strategic manipulation was mathematically incompatible with certain fairness assumptions. These findings anchored a larger research program that explored the logical trade-offs inherent in any decision procedure.

By the 1980s, scholars such as Donald Saari and Michel Balinski introduced geometric and algebraic methods into the analysis of voting rules. These approaches revealed that many well-known paradoxes arise not from particular cases, but from the geometry of the space in which preference profiles reside. The field began to borrow tools from topology, convex geometry, and representation theory, linking social-choice questions to broader developments in pure mathematics.
\end{historical}
