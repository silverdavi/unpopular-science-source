\begin{technical}
{\Large\textbf{Illumina Sequencing and De Bruijn Assembly}}\\[0.3em]

\textbf{Sequencing by Synthesis}\\[0.5em]
Illumina uses reversible terminators with cleavable fluorescent labels. Each cycle:
\begin{align*}
&\text{DNA}_n + \text{dNTP-3'-block-fluor}\\
&\quad\xrightarrow{\text{pol}} \text{DNA}_{n+1}\\
&\text{Imaging} \rightarrow \text{Base identification}\\
&\text{Chemical cleavage} \rightarrow \text{3'-OH restoration}
\end{align*}

Bridge amplification creates clonal clusters ($\sim10^3$ copies) on flow cell surface. Fluorescence signal $S \propto N_{\text{mol}}$ enables base calling with error rate $\varepsilon \approx 0.1\%$.

\textbf{De Bruijn Graph Construction}\\[0.5em]
For read set $\mathcal{R}$ with k-mer length $k$:
\begin{align*}
V &= \{w \in \Sigma^k : w \text{ appears in } \mathcal{R}\}\\
E &= \{(u,v) : \text{suffix}_{k-1}(u)\\
&\quad\quad = \text{prefix}_{k-1}(v)\}
\end{align*}

Each read of length $L$ contributes up to $L-k+1$ distinct vertices (k-mers) and $L-k$ edges along its path, compressing redundant sequence information.

\textbf{Eulerian Path Assembly}\\[0.5em]
Assembly seeks Eulerian path through $G$:
\begin{align*}
\text{Path} &= e_1e_2...e_m \text{ where}\\
&\quad \forall i: \text{head}(e_i) = \text{tail}(e_{i+1})\\
\text{Genome} &= \text{spell}(v_0) + \text{last}(e_1)\\
&\quad + ... + \text{last}(e_m)
\end{align*}

For existence, the underlying graph over nonzero-degree vertices must be weakly connected and either:
- All vertices balanced: in-degree = out-degree, or
- Exactly two semi-balanced vertices (one with out-degree = in-degree + 1 and one with in-degree = out-degree + 1)

\textbf{Coverage and k-mer Selection}\\[0.5em]
Expected k-mer coverage:
\[C_k = C_{\text{read}} \cdot \frac{L-k+1}{L}\]
where $C_{\text{read}} = NL/G$ (reads $\times$ length / genome).

Optimal $k$ balances:
- Small $k$: More connections, higher coverage, ambiguity
- Large $k$: Fewer repeats, lower coverage, gaps

Typically $k \in [21, 127]$ for Illumina data.

\textbf{Graph Complexity}\\[0.5em]
Real graphs contain:
\begin{itemize}[leftmargin=*, topsep=0pt, itemsep=0pt]
\item \textbf{Bubbles}: SNPs/errors create parallel paths
\item \textbf{Tips}: Coverage gaps form dead ends
\item \textbf{Repeats}: Create branching/convergence
\end{itemize}

Error correction: Remove k-mers with coverage $<$ threshold.

\textbf{Paired-End Constraints}\\[0.5em]
Insert size $d \sim \mathcal{N}(\mu, \sigma^2)$ provides scaffolding:
\[|p(r_1, r_2) - \mu| < 3\sigma\]
where $p(r_1, r_2)$ is genomic distance between read pairs.

\vspace{0.5em}
\textbf{References:}\\
{\footnotesize
Bentley et al. (2008). \textit{Nature} 456:53-59.\\
Pevzner et al. (2001). \textit{PNAS} 98:9748-9753.
}
\end{technical}