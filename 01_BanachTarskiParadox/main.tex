A mathematical system begins with a specification of its elements: which objects exist, which operations are defined on them, and which relations must hold. These specifications are encoded in the system's axioms. An axiom is a formal assumption that serves as the foundation for the system. Within that system, no statement can be derived unless it is implied by the axioms in conjunction with the rules of logical inference.

Once a set of axioms is fixed, all further reasoning, including definitions, proofs, and theorems, must proceed within the structure they determine. The consistency and character of the system depend entirely on these initial choices.

Different axiomatic systems describe different mathematical worlds. In one system, every set may have a well-ordering (every nonempty subset has a least element with respect to the order). In another, it may not. In one geometry, parallel lines exist; in another, they do not.

Even familiar objects depend on axiomatic choices. One way to see this is through the von Neumann construction of the natural numbers inside set theory. Think of sets as “bags” that can hold distinct objects: duplicates vanish, so $\{\text{dog},\text{dog},\text{cat}\}=\{\text{dog},\text{cat}\}$. The union of two bags (denoted by $A \cup B$) merges their contents, ignoring duplicates. With this picture, the naturals are nested bags, because all you have in this universe are empty bags. Mark an empty bag as $\{ \}$. 
\[
0:=\{\}, \quad 1:=\{0\}=\{\{\}\}, \quad 2:=\{0,1\}=\{\{\},\{\{\}\}\}, \quad 3:=\{0,1,2\}, \ \dots
\]

Notice the pattern: Zero is an empty bag. One is a bag containing the empty bag. Two is a bag containing both zero and one. Three contains zero, one, and two. Each number $n$ is the bag containing all smaller numbers. This means $m\in n$ (the set $m$ is an element of the set $n$) exactly when $m<n$. The natural numbers are the smallest set that satisfies the axioms.

Addition is defined recursively: $m+0=m$ and $m+S(n)=S(m+n)$, where $S(n)=n\cup\{n\}$ is the successor. Since $n$ is the set of all smaller numbers, $S(n)$ contains all those smaller numbers plus the set $n$ itself as a new element. This makes even the simplest arithmetic fact a theorem rather than a definition. For instance, one proves that $1+1=2$: since $1+1=1+S(0)=S(1+0)=S(1)=2$, the familiar statement is established within the axioms.

This illustrates how set theory provides foundations for all mathematics. Relations like $<$ become sets of ordered pairs, and axioms govern the construction of increasingly abstract objects. Some axioms describe intuitive operations like forming power sets, while others assert the existence of entities that cannot be explicitly constructed, such as inaccessible cardinals or non-measurable sets.

The Axiom of Choice is one such axiom. It asserts that for any collection of non-empty sets, there exists a function that selects exactly one element from each set. In finite cases, such selections can be written down explicitly or proved to exist using elementary methods. In infinite settings, this is not always possible. For example, consider an infinite collection of drawers, each containing a left and a right shoe. A rule such as “choose the right shoe” provides a well-defined selection and does not require the Axiom of Choice. But if each drawer contains a pair of identical socks with no distinguishing features, then no explicit rule can be formulated. The existence of a function that selects one sock from each drawer in this case depends on accepting the Axiom of Choice.

Let us now move down the ladder of abstraction: from axioms to the notion of size. Understanding how we measure things — length, area, volume — will be important for seeing why the Banach-Tarski paradox is so surprising.

How do we assign “size” to things? In everyday life, it's intuitive: a room's area is its length times width, and if you combine two non-overlapping rooms, their total area is simply the sum of their individual areas. 

Measure theory formalizes this intuition. A \emph{measure} $\mu$ is a function that assigns a non-negative real number to certain subsets of a space. The central requirement is \textbf{additivity}: if two disjoint measurable sets $A$ and $B$ are combined, then their measure is the sum of the measures of the two sets: $\mu(A \cup B) = \mu(A) + \mu(B)$ (provided that $A \cap B = \emptyset$, i.e., they share no elements). This principle extends to infinite collections: if a set is decomposed into countably infinite disjoint measurable subsets $\{A_i\}_{i=1}^\infty$, then $\mu\left( \bigcup_{i=1}^\infty A_i \right) = \sum_{i=1}^\infty \mu(A_i)$. This is called \textbf{countable additivity}.

Not every subset can be assigned a measure. Some sets resist consistent size assignment — they are \emph{non-measurable}. The existence of non-measurable sets depends on accepting axioms like the Axiom of Choice.

In familiar settings, the standard measure corresponds to area in the 2d space $\mathbb{R}^2$ or volume in the 3d space $\mathbb{R}^3$. But even in these cases, not all subsets are measurable.

When measuring spatial objects, \textbf{invariance under rigid motions} is required: translating or rotating a measurable set leaves its measure unchanged. Rigid motions preserve distances and angles; they don’t stretch, tear, or compress. This reflects the expectation that volume is an intrinsic property — not dependent on where the object is or how it's oriented.

Non-measurable sets lead to paradoxical results, including the Banach–Tarski paradox. The paradox states that a 3-dimensional ball can be partitioned into five disjoint subsets, which can then be recombined — using only rigid motions — into two balls congruent to the original. This result reflects the failure of volume to be preserved when applied to non-measurable sets. In physical systems such as a stone or a fluid body, each component contributes additively to the whole. The Banach–Tarski construction defines a setting where this principle fails and volume is not additive.

Consider Hilbert's Hotel: an infinite hotel with rooms numbered $1, 2, 3, \ldots$, all occupied. To accommodate one new guest, shift everyone from room $n$ to room $n+1$, freeing room 1. To accommodate infinitely many new guests, move everyone from room $n$ to room $2n$, freeing all odd-numbered rooms.

The Banach–Tarski construction can be presented as a combinatorial game. Consider a deck with four types of cards: $A$, $B$, $A^{-1}$, and $B^{-1}$. These cards can be arranged in sequences, with one rule: if $A$ and $A^{-1}$ appear next to each other, they annihilate. The same applies to $B$ and $B^{-1}$. A sequence like $ABA^{-1}B$ is stable, but a sequence like $B^{-1}BA$ immediately reduces to just $A$. The set of all irreducible sequences forms the free group $F_2$ — all sequences of the cards $A$ and $B$, with the rule that $A$ and $A^{-1}$ annihilate, and $B$ and $B^{-1}$ annihilate.

Now here's where the cards become geometric transformations. Each card corresponds to a rotation of a sphere: $A$ rotates by an irrational angle (e.g., $\sqrt{2}$ degrees) around the Z-axis, $A^{-1}$ rotates back by the same angle. Similarly, $B$ rotates by an irrational angle around the X-axis, $B^{-1}$ rotates back.

Why irrational angles? Because they ensure that no finite sequence of these rotations will ever bring a point back to exactly where it started (unless all the cards cancel out).

When we fix two rotations $A$ and $B$, we can build arbitrary sequences of them and their inverses to move points around the sphere. Pick a point $p$ on the sphere. Apply every possible sequence of $A$, $A^{-1}$, $B$, and $B^{-1}$ to $p$. The resulting collection of points is called the orbit of $p$ under this group of rotations.

Two points lie in the same orbit if one can be turned into the other by some sequence of these rotations. Orbits are disjoint: a point belongs to exactly one orbit, because sequences of rotations either connect two points or they don't. The union of all orbits is the whole sphere, so the orbits form a partition.

Because $A$ and $B$ are chosen carefully (rotations about different axes by irrational angles), the group they generate is free and each orbit is infinite, spreading densely across the sphere.

At this point the Axiom of Choice enters: from each orbit, pick a single representative point. Call the set of these chosen representatives $R$. Every other point on the sphere can be written uniquely as $g \cdot r$, where $g$ is some sequence of rotations and $r \in R$ is the representative of its orbit. In other words, the entire sphere is recovered by “shuffling” the representatives through all possible rotation sequences. 

Now comes the partitioning that makes the paradox work. Four players will divide the entire sphere among themselves by each taking the representatives $R$ and applying only certain rotation sequences:

\begin{itemize}[leftmargin=*]
\item \textbf{Player 1}: Gets all points reached by sequences starting with card $A$ (avoiding immediate $A^{-1}$ cancellation). This creates the point set $S_A$.
\item \textbf{Player 2}: Gets all points from sequences starting with $A^{-1}$ (avoiding $A$), creating $S_{A^{-1}}$.
\item \textbf{Player 3}: Gets all points from sequences starting with $B$ (avoiding $B^{-1}$), creating $S_B$.
\item \textbf{Player 4}: Gets all points from sequences starting with $B^{-1}$ (avoiding $B$), creating $S_{B^{-1}}$.
\end{itemize}

Every possible rotation sequence must start with one of these four cards ($A$, $A^{-1}$, $B$, or $B^{-1}$). This means the four sets are disjoint — no point belongs to two different players — and together they cover the entire sphere (except for fixed points on rotation axes, which are handled separately).

\textbf{First Sphere Reconstruction:}

Player 2 gives their entire collection $S_{A^{-1}}$ to Player 1. Player 1 then rotates every point in $S_{A^{-1}}$ by applying rotation $A$ before each sequence, creating the new set $A \cdot S_{A^{-1}}$.

When you rotate a point reached by sequence $A^{-1}BA$ using rotation $A$, you get $A \cdot (A^{-1}BA) = (AA^{-1})BA = BA$ — the $A$ and $A^{-1}$ cancel. Similarly, rotating the point from sequence $A^{-1}$ gives $A \cdot A^{-1} = $ identity (the original representative point).

Player 1 now has $S_A$ (all points reached by sequences starting with $A$) and $A \cdot S_{A^{-1}}$ (all points reached by sequences NOT starting with $A$). Together, these form a complete sphere!

Meanwhile, Players 3 and 4 perform the same trick: Player 4 gives $S_{B^{-1}}$ to Player 3, who rotates it by $B$ to get $B \cdot S_{B^{-1}}$. Combining $S_B$ and $B \cdot S_{B^{-1}}$ gives another complete sphere.

We started with four disjoint pieces that made up one sphere. By rotating two of those pieces, we reassembled them into two identical spheres. No points were added or removed — only rearranged. That is the heart of the paradox.

This paradoxical outcome depends on selecting representative points from each orbit, which requires a version of the Axiom of Choice. The resulting sets of representatives are non-measurable — they cannot be assigned consistent volumes that preserve both countable additivity and invariance under rotation. The non-measurability arises from the paradoxical nature of the group action itself (the way these rotations act on the sphere).

This construction illustrates the ideas underlying the Banach–Tarski paradox. The full theorem decomposes a three-dimensional ball into five pieces, which can then be rotated into two balls of the same size as the original. The core concepts — free groups acting on a sphere, orbit decompositions, the role of the Axiom of Choice in selecting representatives, and the resulting non-measurable sets — remain the same.

The (Axiom of) Choice occupies a unique position in modern mathematics. Gödel proved that if ZF is consistent, then so is ZFC. Cohen later proved that if ZF is consistent, then so is ZF with the negation of Choice, establishing that the Axiom of Choice is independent of ZF. This independence — the existence of a statement neither provable nor disprovable from the axioms — was revolutionary. Classical theorems across mathematics rely upon it: “every vector space has a basis” and “the product of compact spaces is compact” are equivalent to accepting Choice. 

Before dismissing Choice to avoid the paradox, know that its absence also permits some freighting results. When you partition a set into disjoint subsets, selecting one representative from each part typically creates an injection from the index set into the original set — the number of parts cannot exceed the number of elements. Without Choice, this fails. In ZF alone, it is consistent that a partition $X = \bigsqcup_{i \in I} B_i$ exists with no injection $I \hookrightarrow X$, so the cardinality $|I|$ need not be bounded by $|X|$. A set can be split into more parts than it has elements. Rejecting Choice trades one counterintuitive result for another. 

\begin{commentary}[Commentary]
This chapter forces a distinction between mathematical and physical reasoning. The Banach–Tarski construction is not a paradox in the sense of contradiction or physical impossibility, but it is a good example of the consequences of adopting the Axiom of Choice, showing that intuitive notions like volume are not preserved across all set decompositions. The result is clean and formally sound, yet incompatible with empirical modeling. That gap — between internally consistent mathematics and physically grounded expectation — illustrates the epistemic boundaries explored throughout this book. As with other chapters that emphasize when simplifications fail (relativity in gold, curvature in gravity, topology in voting), this example shows that what appears insane may instead be a well-posed feature of a chosen formal system.

\end{commentary}

\inlineimage{0.35}{01_BanachTarskiParadox/BANACH1.png}{For the following trick we will require the axiom of choice}