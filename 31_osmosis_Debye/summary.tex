Standard osmosis explanations based solely on water concentration gradients fail to account for measured flow rates that far exceed diffusion limits. The ratio of osmotic permeability to diffusive permeability (Pf/Pd) commonly exceeds 100 in biological systems with aquaporins, while purely diffusive transport would yield a ratio near 1. Mechanical explanations, notably Debye's model, attribute osmosis to pressure gradients arising from solute-membrane interactions rather than simple diffusion. When solutes are excluded by a semipermeable membrane, their momentum cannot transfer across the boundary, creating a localized pressure drop that drives water movement.
