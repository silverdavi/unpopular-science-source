\begin{historical}
In the third century BCE, Euclid proved that there are infinitely many prime numbers. His argument, based on contradiction, became one of the earliest and most enduring examples of a general mathematical method. The search for primes — and for patterns among them — soon followed. Eratosthenes introduced a sieve procedure (an algorithm that filters out composites by eliminating multiples) for enumerating primes. By the time of Diophantus, primes were already recognized as foundational to arithmetic.

In the eighteenth century, Euler showed that the sum of reciprocals of primes diverges (a stronger quantitative refinement of Euclid’s theorem that, in particular, implies infinitude), and he introduced analytic tools that connected primes to infinite products and logarithmic identities. This initiated the study of prime distribution through analytic functions.

In 1859, Bernhard Riemann introduced the zeta function into number theory (a complex analytic function encoding prime information via its Euler product) and conjectured that all its nontrivial zeros lie on the critical line. This hypothesis remains unproven. Riemann’s formulation marked the beginning of analytic number theory — a field that uses tools from complex analysis to study the distribution and density of primes. G. H. Hardy and others developed this perspective further in the early twentieth century.

The study of prime gaps took a more technical turn when Viggo Brun introduced sieve methods in the 1910s (combinatorial procedures for bounding the count of integers with prescribed divisibility). Brun proved that the sum of reciprocals of twin primes converges, implying their overall scarcity, even if they might be infinite in number. Later refinements by Selberg and Bombieri led to the Bombieri–Vinogradov theorem (an average-case version of the Generalized Riemann Hypothesis for arithmetic progressions), which became central to modern sieve theory.

In the early 2000s, Goldston, Pintz, and Yıldırım (GPY) introduced a method for bounding small gaps between primes using weighted sums over admissible tuples (integer patterns that avoid local divisibility obstructions — for example, ${0, 2, 4}$ is not admissible, since modulo 3 it covers all residue classes, whereas ${0, 2, 6}$ is admissible). Their work showed that if primes are sufficiently regular in arithmetic progressions, then bounded gaps should follow. The approach relied on conjectural input — notably the Elliott–Halberstam conjecture (a proposed uniformity result for primes in arithmetic sequences).

In 2013, Yitang Zhang proved that there are infinitely many pairs of primes separated by at most 70 million. His argument used a modified version of the GPY sieve. This was the first proof that bounded prime gaps occur infinitely often, without relying on unproven conjectures.
\end{historical}
