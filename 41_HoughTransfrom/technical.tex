\begin{technical}
{\Large\textbf{Radon, Hough, and the Geometry of Detection}}\\[0.3em]

\textbf{Introduction}\\[0.5em]
Detecting geometric structures in images identifies parameter values $p$ for which shape $c(p)$ is present. Both the Radon and Hough transforms implement this principle — one via continuous integration, the other through discrete accumulation.

\textbf{Transform Duality: Reading vs Writing}\\[0.5em]
Let $I(x)$ be a spatial image and $p \in P$ a parameter vector. The Radon transform is the integral projection:
\[
R(p) = \int_{\mathbb{R}^n} I(x)\, \delta(C(x; p))\, dx.
\]
A \emph{reading} operation: each $p$ queries all $x$ satisfying $C(x; p) = 0$ and accumulates their intensity. The Hough transform performs a \emph{writing} operation. For each $x_0$ where $I(x_0) \ne 0$, it computes all $p$ where $C(x_0; p) = 0$ and increments $H(p)$.

Let $\mathcal{C}_p = \{x \in \mathbb{R}^n \mid C(x; p) = 0\}$ denote the pre-image of shape $p$, and $\mathcal{M}_x = \{p \in P \mid C(x; p) = 0\}$ the image of feature $x$. Then:
\begin{align*}
R(p) &= \int_{\mathcal{C}_p} I(x)\, d\mu(x), \quad \text{(Radon)}\\
H(p) &= \sum_{x \in \mathrm{supp}(I)} \mathbf{1}_{\mathcal{M}_x}(p). \quad \text{(Hough)}
\end{align*}
The former computes the inner product between $I(x)$ and a template over $\mathcal{C}_p$, while the latter constructs a discrete indicator function supported on a union of manifolds.

\textbf{Generalized Template Algebra}\\[0.5em]
Let $C(p,x)$ be a generalized function (Gel'fand), with $C(p,x) = \delta(C(x; p))$. The transform is a Fredholm integral operator:
\[
(\mathcal{L}_C I)(p) = \int_{\mathbb{R}^n} C(p,x)\, I(x)\, dx.
\]
This unifies both transforms with classical template matching. If $C(p,x)$ is shift-invariant, i.e., $C(p,x) = K(x - \phi(p))$, the operator reduces to convolution. This links Radon/Hough methods to Fourier-analytic descriptors.

In $\mathbb{R}^2$, detecting lines in normal form $\rho = x\cos\theta + y\sin\theta$ yields
\[
C(\rho,\theta; x,y) = \delta(\rho - x\cos\theta - y\sin\theta).
\]
Substituting into $\mathcal{L}_C I$ recovers the Radon projection; replacing the integral with discrete summation recovers the Hough vote.

\textbf{Intersection Topology in Parameter Space}\\[0.5em]
Every edge point $x$ defines a manifold $\mathcal{M}_x \subset P$. True structures arise where multiple $\mathcal{M}_x$ intersect. The locus $\bigcap_{x \in S} \mathcal{M}_x$ defines candidate parameters supported by feature set $S$. A geometric analogue of data coherence. Nonzero measure intersections indicate shape presence. Noise and discretization perturb $\mathcal{M}_x$, but persistent intersections indicate robustness.

In the continuous limit, $H(p) \to R(p)$ as $I(x) \to \delta$ on features and sampling density increases. The Hough transform is a discrete approximation to the Radon transform under sparse feature distributions.

\textbf{Transform-Invariant Detection and Group Actions}\\[0.5em]
Let $\mathcal{G}$ be a Lie group acting on image space via $\phi_g: \mathbb{R}^n \to \mathbb{R}^n$. A template $T$ transforms covariantly if
\[
C(p, \phi_g(x)) = C(g^{-1} \cdot p, x).
\]
Equivariant detection: shape presence is preserved under group action. Both transforms extend to such scenarios. Detecting rotated ellipses or scaled spirals requires reparameterizing the kernel over the associated group orbit.

\vspace{0.5em}
\textbf{References:}\\[0.5em]
{\footnotesize
M. van Ginkel, C. L. Luengo Hendriks, and L. J. van Vliet (2004). \textit{Introduction to the Radon and Hough transforms and how they relate to each other}. TU Delft Technical Report QI-2004-01.\\
Gel'fand, I. M., \& Shilov, G. E. (1964). \textit{Generalized Functions, Vol. I: Properties and Operations}. Academic Press.
}
\end{technical}
