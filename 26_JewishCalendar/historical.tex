\begin{historical}
Pirkei Avot opens with the chain of tradition: “Moses received the Torah from Sinai and transmitted it to Joshua, and Joshua to the Elders, and the Elders to the Prophets, and the Prophets transmitted it to the Men of the Great Assembly.”

This transmission of authority defined who could determine Jewish law, including calendar matters. The chain continued through specific named authorities: Shimon the Righteous (one of the last of the Great Assembly), Antigonus of Socho, then paired leaders through the generations — the Zugot (pairs), where one served as Nasi (president) and one as Av Beit Din (head of the court).

The pairs included Yose ben Yoezer and Yose ben Yochanan, Joshua ben Perachya and Nittai of Arbel, Judah ben Tabbai and Shimon ben Shetach, Shemaya and Avtalyon, and finally Hillel and Shammai. From Hillel descended a dynasty of leaders who held the title of Nasi through the destruction of the Second Temple in 70 CE and beyond.

During the Temple period, this leadership controlled calendar determination. The Sanhedrin, with the Nasi presiding, declared new months based on witness testimony and intercalated years to maintain seasonal alignment. Their authority to declare time derived from the biblical verse “These are the appointed seasons of the Lord, which you shall proclaim” — the Hebrew emphasizes “which YOU shall proclaim,” granting human authorities the power to establish sacred time.

After the Temple's destruction in 70 CE, the Sanhedrin reconvened in Yavneh under Rabban Yochanan ben Zakkai, then moved through various Galilean cities: Usha, Shefar'am, Beit She'arim, Sepphoris, and finally Tiberias. Despite lacking a Temple, they maintained calendar authority through the traditional chain of ordination (semicha) that connected each generation back to Moses.

The Roman Empire increasingly restricted Jewish self-governance. Emperor Hadrian outlawed ordination after the Bar Kokhba revolt (132-135 CE). Constantine I (306-337 CE) further limited Jewish courts' jurisdiction.

Hillel II served as Nasi from approximately 320 to 385 CE. Facing intensifying persecution and the imminent collapse of centralized Jewish authority, he made an unprecedented decision around 358 CE: publish the mathematical secrets of calendar calculation.
\end{historical}