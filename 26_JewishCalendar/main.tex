Jewish law marks each new day at sunset. This convention creates practical problems at extreme latitudes where the sun remains visible for months, or in orbit where astronauts experience 16 sunsets daily. These edge cases test the boundaries of calendar law developed for Mediterranean latitudes.

The Jewish calendar combines lunar months with solar years. Each month begins with the new moon — the molad — occurring every 29 days, 12 hours, 44 minutes, and 3⅓ seconds. Twelve such months fall short of a solar year by about eleven days. Left uncorrected, holidays would drift through the seasons: Passover in winter, Sukkot in summer. The calendar adds seven leap months over each 19-year cycle, using the correspondence that 235 lunar months approximately equal 19 solar years.

During the Temple era, witnesses who observed the crescent moon testified before the Sanhedrin in Jerusalem. Signal fires transmitted the declaration from mountaintop to mountaintop. Witnesses could lie, clouds could obscure visibility, and distant communities received delayed notification.

The Sanhedrin determined calendar matters. The Nasi (president) presided over seventy sages who determined not just legal matters but calendar matters. Their declaration of the new moon established the month, independent of astronomical observation. This authority allowed practical adjustments when circumstances required.

When Hillel II published the calendar's mathematical rules previously guarded by the Sanhedrin, distant communities could now calculate dates independently. 

The Talmud records a calendar dispute around 70 CE. Two witnesses appeared before Rabban Gamliel claiming they saw the new moon in the morning in the east and the evening in the west — astronomically impossible testimony, which Rabbi Dosa ben Harkinas said: "How can they testify that a woman gave birth, and the next day her belly is between her teeth (still pregnant)?". Rabbi Yehoshua and Rabbi Dosa ben Harkinas declared them false witnesses.

Rabban Gamliel accepted their testimony anyway.

This affected all subsequent holiday dates. If Rabban Gamliel was wrong, then Rosh Hashanah occurred on the wrong day, making Yom Kippur fall on the wrong day ten days later. Rabbi Yehoshua calculated the correct dates according to his understanding and prepared to observe them.

Rabban Gamliel then ordered Rabbi Yehoshua to appear before him "with your staff and your wallet" on the day Rabbi Yehoshua calculated as Yom Kippur. Carrying objects violates the holy day's restrictions. Rabban Gamliel demanded public desecration of what Rabbi Yehoshua believed was the holiest day of the year.

Rabbi Akiva explained to Rabbi Yehoshua: "Whatever Rabban Gamliel has done is valid, for it says, 'These are the appointed seasons of the Lord, holy convocations, which you shall proclaim in their appointed seasons.' Whether in their proper time or not in their proper time, I have no appointed seasons other than these."

Rabbi Dosa ben Harkinas stated: "If we come to question the court of Rabban Gamliel, we must question every court that has arisen from the days of Moses until now." Authority continuity took precedence over astronomical accuracy.

Rabbi Yehoshua took his staff and wallet and walked to Yavneh on his calculated Yom Kippur. When he arrived, Rabban Gamliel stood, kissed him, and declared: "Come in peace, my teacher and my student — my teacher in wisdom and my student because you accepted my words."

The Oven of Akhnai dispute, though not calendar-related, established similar principles of authority. The sages debated whether a particular oven (broken and repaired) could become ritually impure. Rabbi Eliezer ben Hyrcanus argued it could not, offering every possible proof. The other sages disagreed.

Rabbi Eliezer called for supernatural confirmation: a carob tree uprooted itself, a stream flowed backward, the walls of the study house began to fall. Each time the sages responded: "We do not derive law from trees, from streams, from walls."

Finally, Rabbi Eliezer demanded: "If the law is as I say, let it be proven from Heaven!" A divine voice proclaimed: "Why do you dispute with Rabbi Eliezer, seeing that in all matters the law agrees with him?"

Rabbi Yehoshua rose and declared, citing the biblical verse, "It is not in heaven" (Deuteronomy 30:12).

The Talmud reports God declaring: "My children have defeated Me, My children have defeated Me!" The law belongs to human authorities interpreting through human reason. God yields to the rabbinic court's majority decision.

The Ben Meir controversy of 921-922 CE tested whether human consensus could maintain unified practice. By then, Jewish authority had shifted from the Holy Land to Babylon, where the academies of Sura and Pumbedita had become centers of Jewish learning. Aaron ben Meir, claiming authority as a Tiberian scholar in the Holy Land, challenged this Babylonian dominance through calendar calculation.

Ben Meir introduced a new rule (claiming to learn it from his Rabbinic mentors): the molad threshold should be 642 parts after noon (about 35⅔ minutes) rather than the traditional calculation. For the year 922, this meant Passover would fall two days earlier than the Babylonian calculation. This technical dispute meant different communities would observe holidays on different dates.

Ben Meir asserted that proximity to Jerusalem granted special calendar authority. His calculation might have reflected Jerusalem time versus Babylonian time — the 642 parts corresponding to the longitude difference between the two centers, or about questions of exact date of the Creation. But it was in fact less about the calendar and more about authority, challenging the Babylonian academy's authority to determine Jewish law in opposition to the Holy Land's leadership.

Saadia Gaon, head of the Sura academy, wrote mathematical refutations, gathered support from Jewish communities, and challenged Ben Meir. The exilarch (leader of the Jewish diaspora, Reish Galuta) David ben Zakkai and the Babylonian academies excommunicated Ben Meir. Circular letters warned communities against following his calculations. Division over calendar meant division of the people.

Saadia's position prevailed. Modern astronomical calculations place the molad for Tishrei 922 at Saadia's calculated time. His position prevailed because unified practice took precedence over regional authority claims.

Modern geography creates new calendar challenges. Rabbi Yisrael Lipschitz, writing from Danzig in the 1850s, addressed communities in the far north where summer nights never fully darken. "During June and July," he observed, "the night shines like day. At the very least, even at midnight, one can clearly distinguish between tekhelet and white." 

Traditional law uses the ability of our eyes to distinguish between blue and white threads to mark dawn prayers. Continuous visibility eliminates this marker. Rabbi Lipschitz rejected suggestions to estimate based on spring or autumn patterns, noting that communities observed dawn prayers on Shavuot "immediately at dawn," not at estimated times.

At the poles, more extreme conditions apply. "What about someone who comes in summer near the North Pole, where for several continuous months it is actual daytime? There the sun circles the full horizon from east to south to west to north. How should a Jew who arrives there — along with sailors who go there to hunt giant whales — determine his prayer times and Shabbat?"

Rabbi Lipschitz proposed treating each complete sun-circle as one day. If you arrive on Sunday, count seven sun-circles to Shabbat. This solution maintains the seven-day cycle even when "day" loses conventional meaning. But he acknowledged deeper problems: when people at the pole can simultaneously observe the sun with Europeans beginning Shabbat and Americans still in Friday afternoon, which temporal reality governs?

He concluded: "May the Holy One, Blessed Be He, enlighten our eyes with the light of His Torah." This acknowledges the limits of applying Mediterranean-based law to less common conditions.

Modern transportation forced confrontation with global date boundaries. The Chazon Ish (Rabbi Avraham Yeshaya Karelitz) calculated the halakhic date line at 90° east of Jerusalem — approximately 125.2°E longitude — rather than the International Date Line's 180° from Greenwich.

This placed Japan on Sunday when locals observed Saturday. The line would bisect eastern Russia, China, and Australia. To avoid splitting cities, he ruled the 125.2°E meridian curves around land masses, following water.

Most communities rejected this calculation, maintaining local Saturday as Shabbat. Travelers observe stringencies from minority opinions. Practice preserves unity over theoretical precision.

Theory became crisis during the Holocaust. Thousands of yeshiva students from Mir and other Lithuanian centers fled through Siberia to Kobe, Japan, and later to Shanghai. In Kobe — east of the Chazon Ish's calculated line — his view implied Shabbat would fall on Sunday while the local community observed Saturday. As Yom Kippur approached, they cabled desperately for guidance. Rabbi Herzog convened authorities who ruled for local practice. The Chazon Ish telegrammed back: "Eat Wednesday, fast Thursday, fear nothing." In Shanghai — west of his line — Shabbat was kept on Saturday with the established community. Survivors described agonizing between halakhic theory and communal cohesion. 

Orbital flight creates additional complications. Jewish astronauts orbit Earth every 90 minutes, experiencing 16 sunsets daily. Ilan Ramon on Space Shuttle Columbia chose to follow Cape Canaveral time, his last Earth residence. Judith Resnik lit electronic Shabbat candles according to Houston time. These choices reflect the same principle established by Rabban Gamliel: human decision creates sacred time when natural markers fail.

The principle "it is not in heaven" establishes that human authorities interpret law for practical circumstances. Rabbi Yehoshua's compliance with Rabban Gamliel prioritized communal unity over personal calculation. Saadia Gaon's victory over Ben Meir maintained unified practice against regional authority claims. Contemporary rulings for astronauts apply these same principles to orbital conditions.
