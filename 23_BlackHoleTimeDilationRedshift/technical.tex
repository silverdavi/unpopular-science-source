\begin{technical}
{\Large\textbf{Coordinate Reversal and Causal Geometry}}\\[0.1em]

\noindent\textbf{Schwarzschild Metric and the Horizon}\\[0.5em]
The Schwarzschild metric describes spacetime outside a static mass \(M\). In spherical coordinates \((t, r, \theta, \phi)\), it takes the form:
\begin{align*}
\mathrm{d}s^2 =\ 
& -\left(1 - \frac{2GM}{rc^2}\right)c^2\,\mathrm{d}t^2 \nonumber \\
& + \left(1 - \frac{2GM}{rc^2}\right)^{-1}\mathrm{d}r^2 \nonumber \\
& + r^2\,\mathrm{d}\theta^2 + r^2\sin^2\theta\,\mathrm{d}\phi^2.
\end{align*}

The Schwarzschild radius is defined as:
\[
r_s = \frac{2GM}{c^2}.
\]
At \(r = r_s\), \(g_{tt} = 0\), \(g_{rr} \to \infty\). This signals a breakdown in coordinates, not in geometry.

\vspace{0.3em}
\noindent\textbf{Time Dilation and Gravitational Redshift}\\[0.5em]
A static observer at fixed radius \(r > r_s\) experiences proper time:
\begin{equation}
\mathrm{d}\tau = \sqrt{1 - \frac{2GM}{rc^2}}\,\mathrm{d}t.
\end{equation}
As \(r \to r_s\), \(\mathrm{d}\tau/\mathrm{d}t \to 0\). Distant clocks appear to tick normally, but local clocks slow near the horizon.

Photons emitted at \(r_{\text{em}}\) and received at \(r_{\text{obs}} \to \infty\) undergo redshift:
\begin{equation}
1 + z = \left(1 - \frac{2GM}{r_{\text{em}}c^2}\right)^{-1/2}.
\end{equation}
As \(r_{\text{em}} \to r_s\), \(z \to \infty\). Light from near the horizon becomes infinitely stretched and fades from view.

\vspace{0.3em}
\noindent\textbf{Finite Infall and Apparent Freezing}\\[0.5em]
A freely falling observer released from rest at radius \(r_0 > r_s\) reaches radius \(r \le r_0\) in proper time:
\begin{equation}
\tau(r; r_0) = \frac{2}{3}\,\frac{r_0^{3/2} - r^{3/2}}{\sqrt{2GM}}.
\end{equation}
For any finite \(r_0\), the proper time to reach the horizon is finite. The infaller feels no discontinuity; the horizon is not a physical surface. An infaller starting from rest at infinity accumulates infinite proper time before arrival, but the proper time from the horizon to \(r=0\) is finite.

However, to a distant observer, the infaller appears to freeze at \(r = r_s\), due to the divergence of coordinate time: $
t(r) \to \infty \quad \text{as} \quad r \to r_s$.

\vspace{0.3em}
\noindent\textbf{Coordinate Inversion Below the Horizon}\\[0.5em]
Inside the horizon (\(r < r_s\)), the signs of metric components reverse in Schwarzschild coordinates: $g_{tt} > 0, \qquad g_{rr} < 0$.
In this coordinate choice \(r\) behaves timelike and \(t\) spacelike; in horizon-regular coordinates (e.g., Eddington–Finkelstein, Kruskal) this interpretation is seen as a coordinate effect while causal futures still point to decreasing \(r\).

The physical implication is that movement in \(r\) becomes mandatory. All future-directed timelike paths lead to smaller \(r\), ending at the singularity \(r = 0\). The singularity is not “a place inside” but a moment in the infaller’s proper future.

This inversion  reflects the geometry. Any attempt to "hover" or remain at constant \(r\) is no longer physically possible once inside the horizon.

\vspace{0.3em}
\noindent\textbf{Light Cones and Irreversibility}\\[0.5em]
At the horizon, outgoing light rays remain on the surface: $\frac{\mathrm{d}r}{\mathrm{d}t} = 0 \quad \text{for outgoing null rays at } r = r_s$.

Below the horizon, all light cones tip inward. Future lightlike and timelike paths are directed toward decreasing \(r\). There is no direction within the cone that leads to increasing radius.

This defines the event horizon as a one-way temporal boundary: a surface from which causal influence cannot escape outward.

\vspace{0.7em}
\noindent\textbf{References:}\\
{\footnotesize
Misner, C. W., Thorne, K. S., Wheeler, J. A. (1973). \textit{Gravitation}.\\
Carroll, S. M. (2004). \textit{Spacetime and Geometry}.\\
Wald, R. M. (1984). \textit{General Relativity}.
}
\end{technical}
