\begin{historical}
Arnold Sommerfeld’s 1916 work on relativistic extensions to the Bohr model laid the groundwork for understanding how high nuclear charge alters electronic structure in heavy atoms. In 1940, AO Williams improved the Hartree self-consistent field method by incorporating the Dirac equation, demonstrating relativistic corrections in Cu$^+$ and quantifying the spin–orbit splitting — the splitting of degenerate energy levels due to coupling between an electron’s spin and its orbital motion.

David Francis Mayers extended this work in 1957, identifying that electrons in heavy atoms, traveling at significant fractions of the speed of light, experience orbital contraction.

Boyd, Larson, and Waber expanded upon this by demonstrating the relativistic expansion of certain d orbitals, emphasizing the intricate interplay between electron velocity and orbital behavior. Kenneth S. Pitzer’s 1971 research marked a turning point, showing that mercury’s unusually low melting point could be attributed to these relativistic effects. Later in the 1970s, Pekka Pyykkö and Jean-Pierre Desclaux carried the idea further, using theoretical methods to connect mercury’s liquid state and gold’s distinct coloration to changes in orbital energies brought about by relativistic corrections.

By the early 1980s, X-ray photoelectron spectroscopy offered direct experimental confirmation of these phenomena, although Lennart Norrby noted in 1991 that such insights still struggled to gain widespread inclusion in general chemistry curricula.
\end{historical}
