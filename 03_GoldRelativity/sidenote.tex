\begin{SideNotePage}{
  \textbf{Energy–Mass Equivalence} \par
  A box of mass \(M\) and length \(L\) emits a light pulse of energy \(E\) from its left wall. The pulse carries momentum \(p = E/c\), so to conserve momentum the box recoils leftward with speed \(v \approx E/(M c)\) (for \(v \ll c\)). The pulse reaches the right wall in time \(t \approx L/c\), during which the box drifts left by \(\Delta x = v t = \frac{E L}{M c^2}\). After absorption the box is again at rest, but its location has shifted—seemingly moving the system’s center of energy despite no external forces.

  To prevent any net shift, the emission must reduce the box’s rest mass at the left wall by \(\Delta M\) (which reappears at the right upon absorption). This relocation shifts the system’s center of energy rightward by \(\Delta x' = \frac{\Delta M}{M} L\). Setting \(\Delta x' = \Delta x\) gives \(\frac{\Delta M}{M} L = \frac{E L}{M c^2}\), hence \(\Delta M = E/c^2\).
 
  \vspace{0.5em}
  \textbf{Relativistic Energy and Momentum} \par
  In the middle-right panel, a massive particle crosses the box; both rest mass and motion contribute. Lorentz symmetry packages energy and momentum into the four‑momentum \(P^\mu = (E/c, p_x, p_y, p_z)\). Its Minkowski norm is invariant: \(P^\mu P_\mu = (E/c)^2 - p^2 = m^2 c^2\). Therefore \(E^2 = m^2 c^4 + p^2 c^2\) (or, with \(c=1\), \(E^2 = m^2 + p^2\)).

  \vspace{0.5em}
  \textbf{Why Gold Is Yellow} \par
  In the final panel, blue light is absorbed at the surface of a gold atom. Relativistic contraction of inner orbitals (due to high-velocity 6s electrons) shifts energy levels. This narrows the 5d–6s gap, bringing blue transitions into range. The missing blues tint the reflection yellow. Gold is yellow because relativity bends its atomic spectrum.
}{03_GoldRelativity/03_ All That Gold Glitters.pdf}
\end{SideNotePage}
