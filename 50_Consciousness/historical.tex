\begin{historical}
Ether-era anesthesia began in 1846 with Morton’s public demonstration in Boston; within months, ether and chloroform spread worldwide. By the early 20th century, Meyer and Overton independently observed a correlation: anesthetic potency scaled with lipid solubility across diverse compounds. This supported the idea that consciousness could be turned off by a nonspecific action on neuronal membranes. Yet the correlation cracked under scrutiny: highly lipophilic yet inert molecules failed to anesthetize, while effective agents deviated from the predicted potency.

Mid-to-late 20th century work shifted toward specific molecular targets. Volatile agents were shown to prolong inhibitory currents at GABA\textsubscript{A} receptors, while nitrous oxide and ketamine disrupted glutamatergic signaling via NMDA antagonism. Parallel findings implicated two-pore K\textsuperscript{+} (K2P) channels and hyperpolarization-activated cyclic nucleotide–gated (HCN1) currents in setting neuronal excitability under anesthetics. Still, no single pathway unified the class.

In prion disease, a different historical thread exposed the opposite failure mode. In 1982, Prusiner proposed prions — proteinaceous infectious particles — as agents of neurodegeneration. A rare PRNP mutation producing fatal familial insomnia (FFI) was later traced to selective thalamic degeneration, abolishing sleep despite otherwise preserved wakeful function. An Italian pedigree provided the defining clinical arc: onset with fragmented sleep, inexorable insomnia, autonomic failure, cognitive collapse, and death within months. Where anesthesia induced obliviousness, FFI prevented it.

\end{historical}
