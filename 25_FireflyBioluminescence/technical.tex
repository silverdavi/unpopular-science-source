\begin{technical}
{\Large\textbf{Bioluminescence Mechanics and Quantification}}\\[0.2em]

\noindent\textbf{Molecular Reaction}\\
Firefly luciferase catalyzes ATP-driven luciferin oxidation, producing excited oxyluciferin that emits a photon at $\sim 560\,\text{nm}$ ($E \approx 3.55 \times 10^{-19}\,\text{J}$) with quantum yield $\sim 0.4$. Flash duration (200–300 ms) is controlled by oxygen availability via tracheal gating to photocytes.

\noindent\textbf{Bottom-Up Biochemical Calculation}\\
Photon emission from enzyme abundance and oxygen-limited kinetics ($10^6$ molecules/photocyte $\times$ $10^5$ photocytes $= 10^{11}$ total luciferase):
\begin{align*}
\text{Reaction rate} &= 10^{11}\,\text{enzymes} \times 0.01\,\text{s}^{-1} = 10^9\,\text{rxn/s} \\
\text{Photon flux} &= 10^9 \times 0.4 = 4 \times 10^8\,\text{photons/s} \\
\text{Photons/flash} &= 4 \times 10^8 \times 0.25\,\text{s} \approx 10^8\,\text{photons}
\end{align*}
Flash intensity is limited by oxygen delivery, not enzyme saturation. In vivo turnover (0.01–0.1 s$^{-1}$) is orders of magnitude below in vitro $k_\text{cat}$ maximum (1–2 s$^{-1}$). This prediction is independently corroborated by modern spectrometry (Goh \& Wang, 2022: $3$–$5 \times 10^8$ photons/flash) and expert estimates (T.R. Fallon, pers. comm.: $10^8$–$10^9$).

\noindent\textbf{Historical Discrepancy}\\
The biochemical prediction ($\sim 10^8$ photons/flash $\approx 7.6 \times 10^{-9}$ candle) contradicts the canonical measurement from Ives \& Coblentz (1924) — propagated through textbooks for a century — of 1/40 candle (0.025\,cd, implying $\sim 3 \times 10^{14}$ photons/flash). This six-order-of-magnitude discrepancy likely reflects methodological limitations of early photometry and uncritical propagation. The biologically plausible range, constrained by enzyme abundance, oxygen delivery, and ATP budgets, is $\sim 10^8$–$10^{10}$ photons/flash. Firefly light appears brighter to human vision than to cameras due to near-perfect overlap with peak photopic sensitivity (555\,nm).
\noindent\textbf{References:}\\
{\footnotesize
Wilson, T. \& Hastings, J.W. (1998). Bioluminescence. \textit{Annu. Rev. Cell Dev. Biol.}, \textbf{14}, 197-230.\\
Shimomura, O. (2012). \textit{Bioluminescence: Chemical Principles and Methods}. World Scientific.\\
Ives, H. E., \& Coblentz, W. W. (1924). Photometric studies of luminous insects. \textit{J. Opt. Soc. Am.}, 9(3), 217–236.
}
\end{technical}