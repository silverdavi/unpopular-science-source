\begin{technical}
{\Large\textbf{The Phonological Evolution of Labiovelars in the Indo-European Descendants of \piefont{*kʷékʷlos}}}\\[0.3em]

Proto-Indo-European (PIE) contained labiovelar stops (\piefont{*kʷ}, \piefont{*gʷ}, \piefont{*gʷʰ}) with simultaneous velar closure and labialization, contrasting with plain velars (\piefont{*k}, \piefont{*g}, \piefont{*gʰ}) and palatalized velars (\piefont{*ḱ}, \piefont{*ǵ}, \piefont{*ǵʰ}). The PIE root \piefont{*kʷékʷlos}, a reduplicated form of \piefont{*kʷel-} ("to turn"), underwent systematic shifts across branches.

\textbf{In Greek},
By Mycenaean Greek (c. 1400 BCE, attested in Linear B, the earliest known form of the Greek writing system), labiovelars were still distinguished. Later Greek neutralized them context-dependently: \piefont{*kʷ} became \piefont{t} before front vowels (\textgreek{πέντε} "five" < PIE \piefont{*pénkʷe}), \piefont{p} in many environments (\textgreek{λείπω} "I leave" < PIE \piefont{*leikʷ-}), and \piefont{k} in certain contexts:
\[
\textgreek{κύκλος} \text{ (\emph{kyklos}) < PIE } \piefont{*kʷékʷlos}
\]
The labiovelars were thus fully neutralized, with reflexes depending on phonological environment.

\textbf{In Sanskrit}, labiovelars merged with palatals before front vowels, so \piefont{*kʷ} became \textsanskrit{च} (\emph{c}, [t͡ʃ]):
\[
\textsanskrit{चक्र} \text{ (\emph{chakra}) < PIE } \piefont{*kʷékʷlos}
\]
This is part of a broader Indo-Iranian shift where labiovelars fronted or merged with palatals.

\textbf{In Latin}, \piefont{*kʷ} was generally preserved as \emph{qu} in most environments (\emph{quis}, \emph{quo}, \emph{equus}, \emph{aqua}, \emph{quattuor}). However, certain roots show regular de-labialization to \emph{c-}, notably the \piefont{*kʷel-} family: PIE \piefont{*kʷel-} → Latin \emph{colere} ("to cultivate"), \emph{incola} ("inhabitant"). This reflects dissimilation: the labial element of \piefont{*kʷ} is lost before a rounded vowel in the following syllable.
In contrast, \emph{circulus} ("circle"), from Greek \emph{kirkos} ("ring"), derives from PIE \piefont{*(s)ker-} "to turn, bend" — a distinct root without labiovelars.

\textbf{In Proto-Germanic}, Grimm's Law altered the stop system: \piefont{*kʷ} → \piefont{*hw}. Thus, \piefont{*kʷékʷlos} became \piefont{*hweulą} (Proto-Germanic), which evolved into Old English \emph{hwēol}, Middle English \emph{whele}, and Modern English \emph{wheel}.

\textbf{Summary}: Greek neutralized labiovelars context-dependently (\piefont{*kʷ} > \piefont{t/p/k}), Sanskrit palatalized (\piefont{*kʷ} > \piefont{c} → \textsanskrit{चक्र}), Latin generally preserved \emph{qu} but de-labialized in certain roots like \emph{colere}, and Germanic fricativized via Grimm's Law (\piefont{*kʷ} > \piefont{hw} → \emph{wheel}).

\noindent
These transformations illustrate how a single PIE labiovelar stop produced diverse reflexes across Indo-European languages, shaping words that remain etymologically linked despite significant phonetic divergence.

\vspace{0.5em}
\noindent\textbf{References:}\\
{\footnotesize
Fortson, B. (2010). \textit{Indo-European Language and Culture: An Introduction}. Wiley-Blackwell.\\
Ringe, D. (2006). \textit{From Proto-Indo-European to Proto-Germanic}. Oxford University Press.\\
Online Etymology Dictionary: \url{https://www.etymonline.com/}\\
}
\end{technical}
