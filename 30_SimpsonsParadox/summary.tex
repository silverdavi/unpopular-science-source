Simpson's Paradox occurs when a statistical trend present in separate groups reverses when the groups are combined. This effect is a result of unequal group sizes or hidden confounding variables that distribute non-uniformly across the data. For example, a treatment might show positive effects in both male and female subgroups yet appear harmful in the aggregate population if the treatment is disproportionately given to patients with more severe conditions and males and females differ in average severity. The apparent paradox demonstrates that causal inference requires careful consideration of the causal relationship rather than relying solely on raw correlations.
