\begin{SideNotePage}{
  \textbf{Top (Curse of Dimensionality – Vanishing Volume of the Sphere):}  
  As dimensionality increases, the volume of an $n$-sphere relative to the surrounding $n$-cube rapidly shrinks toward zero. In low dimensions, the sphere fills most of the cube; by around 10 dimensions, it's practically gone. This shows that intuition from 2D or 3D fails in high dimensions — most volume concentrates in the corners. \par

  \textbf{Bottom (Blessing of Dimensionality – Human Uniqueness in High-D Spaces):}  
  If each person is described by even 50 independent traits (drawn from uniform or Gaussian distributions), then the "average" human lies in a vanishingly small region of space. Almost everyone is in the high-dimensional fringes — radically unique combinations of attributes. High dimensionality ensures that individuality is not rare but inevitable. \par
}{33_IncubationInequality/33_ Gaussian Correlation Inequality.pdf}
\end{SideNotePage}
