A geometric puzzle about Gaussian probability stumped mathematicians for over 60 years: prove that convex sets that are symmetric around the origin have enhanced overlap under Gaussian measure — that P(A ∩ B) ≥ P(A) · P(B). Despite partial results for boxes, ellipsoids, and slabs, the general case resisted all attempts. In 2014, Thomas Royen, a retired pharmaceutical statistician from a small German university, solved it using textbook methods: transforming to squared variables, applying Laplace transforms, and checking matrix determinants. His proof, published in an obscure journal, went unnoticed for years. 
