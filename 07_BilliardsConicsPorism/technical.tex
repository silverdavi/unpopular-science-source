\begin{technical}
\textbf{\Large Poncelet’s Theorem for Two Ellipses}\vspace{0.2em}

Let \( C \) and \( D \) be two nested ellipses in the plane, with \( D \subset \operatorname{int}(C) \). The \emph{Poncelet map} \( T: C \to C \) is defined as follows: for a point \( p \in C \), draw the line through \( p \) tangent to the inner ellipse \( D \); let \( T(p) \) be the second point of intersection of this line with \( C \), following a fixed orientation. For \( C \) and \( D \) this defines an orientation-preserving \( C^\infty \) diffeomorphism of \( C \).

We show that \( T \) preserves a natural measure and is topologically conjugate to a circle rotation. This yields a complete classification of the dynamics of \( T \), and with it, a proof of Poncelet's closure theorem.

\textbf{Affine Reduction and the Invariant Measure}

To analyze \( T \), we apply an affine transformation to simplify the geometry. Suppose \( C \) is given by
\[
{x^2}/{a^2} + {y^2}/{b^2} = 1, \quad \text{with } a > b > 0.
\]
Let \( U(x, y) = (x, \frac{a}{b} y) \), a linear map sending \( C \) to a circle \( C' \), and \( D \) to an ellipse \( D' \). Let \( \tilde{p} = U(p) \), and let \( \tilde{s} \) denote arc-length on \( C' \). Let \( m \) be the tangency point of the chord from \( p \) to \( T(p) \) with \( D \), and set \( \tilde{m} = U(m) \). Define the tangent-chord distance \emph{after} the affine map: $\tilde{\rho}(\tilde{p}) := |\tilde{p} - \tilde{m}|$. The invariant measure on \( C' \) is $d\mu(\tilde{p}) := {d\tilde{s}(\tilde{p})}/{\tilde{\rho}(\tilde{p})}$, which pulls back to \( C \).

This measure captures the rate of change of arc-length with respect to tangency.

\textbf{Invariance of the Measure.} Work on \( C' \): for nearby \( \tilde{p}, \tilde{p}' \in C' \), let \( \tilde{T}(\tilde{p}) = \tilde{p}_1 \), \( \tilde{T}(\tilde{p}') = \tilde{p}_1' \). The chords \( \tilde{p}\tilde{p}_1 \) and \( \tilde{p}'\tilde{p}_1' \) intersect at \( \tilde{n} \). On the \emph{circle}, by pole-polar duality in projective geometry, one obtains:
\[
\frac{|\tilde{p}_1' - \tilde{p}_1|}{|\tilde{p}' - \tilde{p}|} = \frac{\tilde{\rho}(\tilde{p}_1)}{\tilde{\rho}(\tilde{p})}.
\]
Taking the limit as \( \tilde{p}' \to \tilde{p} \):
\[
\frac{d\tilde{s}_1}{d\tilde{s}} = \frac{\tilde{\rho}(\tilde{T}(\tilde{p}))}{\tilde{\rho}(\tilde{p})},
\]
hence \( {d\tilde{s}}/{\tilde{\rho}} \) is invariant. Pulling back to \( C \), \( \mu \) is preserved by \( T \).

\textbf{Topological Conjugacy to a Circle Rotation}
An orientation-preserving circle homeomorphism with a finite invariant measure positive on every nonempty open arc has no wandering intervals; therefore its Poincaré semiconjugacy to a rotation is a conjugacy.

Our measure \( \mu \) satisfies:\\
\textbf{Finite:} Since \( D \subset \operatorname{int}(C) \) are smooth and strictly convex, \( \tilde{\rho} \) has a positive infimum on \( C' \), so \( \int_{C'} d\tilde{s}/\tilde{\rho} < \infty \). \textbf{Non-atomic:} points have zero measure. \textbf{Positive on arcs:} both \( \tilde{\rho} \) and arc-length are positive.

Thus, there exists a homeomorphism \( \varphi: C \to S^1 \) and \( \alpha \in [0,1) \) such that:$ \varphi \circ T \circ \varphi^{-1} = R_\alpha, \quad \text{where } R_\alpha(\theta) = \theta + \alpha \mod 1$

The number \( \alpha \), called the \emph{rotation number} of \( T \), quantifies the average angular displacement per iteration. Lift the circle map to \( F: \mathbb{R} \to \mathbb{R} \) in an angular coordinate; then $\alpha := \lim_{n \to \infty} \frac{F^n(\theta) - \theta}{n} \mod 1$. This limit exists and is \( p \)-independent. Equivalently, after normalizing \( \mu(C)=1 \), the map is a rigid rotation in that coordinate, so \( \alpha=\mu([p,T(p))) \) (mod 1) (independent of \( p \)).

The rotation number classifies the dynamics:
\begin{itemize}
  \item If \( \alpha \in \mathbb{Q} \), say \( \alpha = \frac{m}{n} \), then all orbits are periodic with period \( n \). Every point on \( C \) traces a closed \( n \)-gon tangent to \( D \).
  \item If \( \alpha \notin \mathbb{Q} \), then no orbit is periodic. The sequence \( p, T(p), T^2(p), \ldots \) becomes dense in \( C \), and no Poncelet polygon closes.
\end{itemize}

\vspace{0.25em}
\textbf{References:}\\
{\footnotesize
Leopold Flatto, \textit{Poncelet's Theorem}. Mathematical Surveys and Monographs, Vol. 56. American Mathematical Society, 2009 (beautiful book!)\\
For visualization, see \href{https://bit.ly/poporism}{\texttt{bit.ly/poporism}}.\\
}
\end{technical}
