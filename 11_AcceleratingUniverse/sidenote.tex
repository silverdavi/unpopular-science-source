\begin{SideNotePage}{
  \textbf{Top (Cosmic Expansion Scenarios):} The universe’s rate of expansion is shown across time, beginning with an early deceleration due to gravity, followed by the observed acceleration linked to dark energy. Possible future outcomes are indicated: the \emph{Big Crunch} (re-collapse), the \emph{Big Freeze} (endless cooling expansion), and the \emph{Big Rip} (runaway acceleration tearing structures apart). Supernova data points provide the evidence anchoring this curve. \par
  \textbf{Bottom (Cosmic Energy Composition):} The universe’s mass-energy content is divided into dark energy (72\%), dark matter (23\%), and atoms (4.6\%), with small additional contributions from neutrinos and photons. Within this narrow slice of atomic matter we can zoom in further: the large-scale structures of the cosmos \(\to\) stars and galaxies \(\to\) planetary systems \(\to\) laboratory beakers. This exponentially droppings equence highlights how minuscule the portion we directly observe is compared with the vast domain over which the same physical laws are successfully extrapolated. \par

}{11_AcceleratingUniverse/11_ Dark Energies Are Pushing Us Apart.pdf}
\end{SideNotePage}
